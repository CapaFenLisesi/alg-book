\chapter{Quadratic Functions}
\label{ch:quadfunc}

\chapquote{Nature creates curved lines while humans create straight lines.}{Hideki Yukawa\\Japanese theoretical physicist}

%\begin{quote}
%Nature creates curved lines while humans create straight lines.
%\par\hfill --- Hideki Yukawa, Japanese theoretical physicist	
%\end{quote}

Every chapter should have a lead paragraph -- even just a short one -- that appears before the first heading. This is a placeholder paragraph which will at some point be replaces by actual content.

Graphing... Standard and Vertex Form...

% % % % % % % % % % % % % % % % % % % % % % % % % % % % % % % % % % % % % % % % 
%\section{Introduction to Quadratic Functions}
%\label{sec:quadintro}
%
%
%
%Vocabulary
%
%Parabola The U-shaped graph of a quadratic function. There is the geometric definition:
%the set of all points the same distance from a fixed point called a focus and a line
%called a directrix.
%
%Vocabulary
%
%Axis of Symmetry The line that passes through the vertex and divides the parabola into two
%symmetric parts.
%
%Vocabulary
%
%Vertex The lowest or highest point on a parabola
%
%Vocabulary
%
%Standard Form of a Quadratic Function A quadratic function in the form y = ax2 + bx + c,
%where a \$ eq\$ 0
%
%
%Vocabulary
%
%Vertex Form A quadratic function in the form y - k = a(x - h)2 or y = a(x - h)2 + k ,
%where (h, k) is the vertex.
%
%Vocabulary
%
%Factored Form A quadratic function in the form y = (ax + b) (cx + d), where a and c
%$\neq$ 0. Related to intercept form which is y = a(x-r1)(x-r2) where r1 and r2 are the
%roots (x-intercepts) of the related quadratic equation.
%
%Vocabulary
%
%Second Difference When a table of values shows domain values that are sequential integers,
%the second difference is the difference of the difference of the independent variable's
%values. The second difference of a quadratic function is constant.
%
%Vocabulary
%
%Quadratic Function A nonlinear function that can be written in the standard form y = ax2 +
%bx + c, where a \$ eq\$ 0
%
%Vocabulary
%
%Maximum Value For y = ax2 + bx + c where a <0, the y-coordinate of the vertex is the
%maximum value of the function.
%
%Vocabulary
%
%Minimum Value For y = ax2 + bx + c where a >0, the y-coordinate of the vertex is the
%maximum value of the function.
%
%Vocabulary
%
%Fixed Perimeter A set perimeter that can't change due to given constraints.
%
%Staking a Claim --> Introduction
%
%Today you begin your study of the last family of functions for Algebra 1, the Quadratic
%Function. By the end of the lesson, you are going to know the characteristics of
%quadratics as they appear in tables, graphs and equations. You are going to use one of the
%classics examples of quadratic relationships to achieve this, the fixed perimeter problem.
%
%Follow-Up From Staking a Claim (Data and Tables)
%
%1. What is the domain of the function represented in the table? * If x = length, (integers
%or real) 0 < x < 20 * In general, quadratics have the domain of all reals
%
%2. What is the range of the function represented in the table? * If y = area, (integers or
%real) 0 < y $\leq$ 100 * In general, quadratics have two ranges Y $\leq$ Maximum
%Y $\geq$ Minimum
%
%3. Should we include length and widths of zero? * Yes, because when I extend the domain to
%include the rational numbers, I need to have those as boundaries.
%
%4. What properties does the table have? * Symmetric --> where does the line of symmetry
%occur? At x = 10 * Increases, hits a maximum area, then decreases --> in general they can
%also decrease, hit a minimum, then increase * The rate at which it increases slows down.
%It is not constant. If it were it would be a line. * In general, if the Independent
%Variable values are sequential, this will be true at some point in the table. You can't
%rely on this ``increases then decreases'' (or the reverse) to determine if a table is
%quadratic. You might not get that part of the table that shows that, so we have to look
%more closely at the pattern
%
%5. What are some of the patterns that you notice? * Increase, hits a maximum value, then
%decreases --> in general, it can also Decrease, hits a minimum value, then increase * Rate
%of change is not constant, but changes in a pattern . The pattern just happens to be
%linear!--> most important part of this!!!! * Second Difference is Constant (and not = to
%zero)--> true for all quadratics and it is the test to determine if a table represents a
%quadratic relationship. This is what creates the familiar curve in the graph. * In
%general, the sign of the second difference will tell if the parabola is going to increase
%then decrease (sad parabola = negative 2nd difference) or if it is going to decrease then
%increase (happy parabola = positive 2nd difference)
%
%6. How do we know that the table is quadratic? * From the pattern activity in the
%beginning of the year, constant second difference
%
%Length Width Area First Diff. Second Diff. 0 20 0 19 17 15 13 11 9 7 5 3 1 -1 -3 -5
%Constantly decreasing rate of change -2 -2 -2 -2 -2 -2 -2 Constant and negative second
%difference.
%
%The table must be sequential to show this. The x's can count by 1, 2, 3, as long as it is
%consistent. 1 19 19
%
%
%2 18 36
%
%
%3 17 51
%
%
%4 16 64
%
%
%5 15 75
%
%
%6 14 84
%
%
%7 13 91
%
%
%8 12 96
%
%
%9 11 99
%
%
%10 10 100
%
%
%11 9 99
%
%
%12 8 96
%
%
%13 7 91
%
%
%Etc.
%
%
%
%
%
%Follow-Up from Staking a Claim (Graphs)
%
%1. What is the domain and range for the graph? --> See above
%
%2. Describe the shape of the graph. * It is a function (passes the vertical line test) *
%Parabola --> U-shaped
%
%3. What are some special features of the graph? * It has a maximum value --> parabola with
%a maximum (aka opens down or sad parabola) * This maximum/minimum is called the Vertex *
%In general, it can also have a minimum value --> parabola with a minimum (aka opens up or
%happy parabola) * The graph is not made up of straight line segments. It is a curve that
%is less steep the closer you get to the vertex. The steepness increases as you move away
%from the vertex. --> true for all parabolas * Has a line of symmetry and the vertex occurs
%at the axis of symmetry --> true for all * Has two x-intercepts that are equidistant from
%the line of symmetry --> true for all that have 2 x-intercepts. In general you can have 0,
%1, or 2 x-intercepts. If there is only one, the vertex is the x-intercept because every
%other time, you get two x's for a single y-value. If you have no x-intercepts, the graph
%starts above or below the x-axis. * There are two y-values for all x-values except for the
%vertex. \ldots{} every point has a buddy on the other side of the line of symmetry except
%for the vertex.
%
%4. What occurs at the axis of symmetry? --> the vertex 5. Can you see these features in
%the table? Yes. There is a line of symmetry, two y's for each x and the rate of change
%dictated the shape of the function
%
%
%
%
%Follow Up From Staking a Claim (Equations)
%
%1. What is one of the equations that will represent this data? And how did you come up
%with it? * Y = x(20-x) * X = length, 20-x = width., multiply length and width to get area
%* This is called the factored form. This shows the relationship as a product. You can only
%have this form if the graph has x-intercepts! --> hint for the future, if we can write a
%quadratic in this form, we can very easily find the x-intercepts * Notice that for fixed
%perimeter problems there is a general formula for the area. . 2. How can I turn this into
%a different equation? * Distribute the x * Y = 20x - x2 or Y = -x2 + 20x * This is called
%the standard form. This is the only form (that we are going to study) that you can use for
%all quadratic functions and that the definition of the Quadratic Function is based on this
%specific equation. * The a, b, and c tell you quite a bit about the graph and we will look
%more closely at this next class. * Notice that for fixed perimeter problems, there is a
%general formula of the area or
%
%3. Is there another form? * Of Course! There are several others. One often used in Algebra
%II is called the vertex form. It is the form that uses the vertex right in the equation.
%
%Follow Up From Staking a Claim ( Using Quadratic Functions to solve problems.)
%
%1. What is the greatest possible area? 100 sq yds.
%
%2. What shape is this specifically? A square
%
%3. How can I use the table, graph, equation to answer this? * You are looking for the
%maximum value! The y-value is the maximum area and the corresponding x-values tells you
%the side length of the square. --> In general, you will often have to find the maximum to
%answer a word problem. Sometimes you want y- or x- intercepts too. * Now, the graph
%actually tells me more about fixed perimeters in general. --> see picture below
%
%4. Does opening up the domain to be real numbers (as opposed to integers) change this
%solution?-->No. The graph is the picture of all such rectangles. The data points in
%between the ones that were graphed are those with non-integer side lengths. So, the
%maximum is still the maximum.
%
%5. What shape would I use if I really wanted to maximize area? --> A circle! The perimeter
%is fixed, no matter what. The more sides you add to that polygon, the bigger the area is
%going to be. So a regular hexagon will have a greater area than a square with the same
%perimeter. If you extend this reasoning, then a regular polygon with infinite sides will
%have the biggest area of them all\ldots{} and that is just a circle!
%
%\section{Graphing Quadratics}
%
%What We Know So Far\ldots{}
%
%Standard Form: y = ax2 + bx + c where
%
%The Graphs Features: * They are U-shaped curves called parabolas. * Parabolas are
%symmetric * The vertex (max or min) will occur on the line of symmetry * They can open
%up(happy) or open down (sad) * They have a max if they open down and a min if the open up
%* They can have 0, 1, or 2 x-intercepts (aka roots or zeroes of the function) * If there
%is one x-intercept, it is the vertex * If there are two x-intercepts, they are equidistant
%from the axis of symmetry.
%
%Transformations (what changing a, b, and c do to the graph)
%
%The Parent Function: y = x2 1. a * If a is positive, the parabola opens up * If a is
%negative, the parabola opens down * If the parabola is the same width as the parent * If
%the parabola will be narrower than the parent * If the parabola will be wider than the
%parent 2. b shifts left or right 3. c * This coordinates of the y-intercept are (0, c), if
%you plug in zero for x, you get c for y. * If a value was just added to or subtracted from
%the parent ( y = ax2 + c) , it would move the parabola up or down
%
%Graphing Strategy
%
%Every unique parabola can be created with exactly 3 points. That's why you need 3 data
%points to do a quadratic regression on a calculator.
%
%So, the bare minimum to graph, or define, a parabola is 3 points. (Similarly, lines needed
%two points). So instead of making huge tables that have many points, we are going to find
%a few strategic points to plot so that we can sketch a reasonable curve efficiently.
%
% Points we want: (1) the vertex --> the most important point!!!!!!!!! (2) the y-intercept
% (3) the x-intercept(s).
%
%Remember, parabolas are symmetric. Take advantage of that!
%
%How to Graph
%
%1. Stop to look at your equation and think about what your graph is going to look like by
%looking at the values of a, b, and c 2. Find and plot the Axis of Symmetry --> this tells
%us where to find the vertex, and gives us a way to double the number of additional points
%to plot. 3. Find the vertex --> It is on the Axis of Symmetry, so substitute in that
%x-value to find the y-coordinate. It tells you range\ldots{} SUPER IMPORTANT POINT 4. Find
%the x- and y-intercepts --> If you have a standard form equation, the y-intercept is easy.
%It is that ``c'' value. Finding the x-intercept, now that is tricky. We'll see why in soon
%enough 5. Get any additional points as necessary to sketch a reasonable curve. --> If you
%can't find the x-intercepts, or if the graph doesn't have any, and you need to find
%another point, just substitute in some value of x into the function to find the coordinate
%of another point.
%
%Graphing of the form y = ax2 + c
%
%What are some features that all these graphs? 1. They all have the y-axis as the line of
%symmetry. 2. Their vertices will be their y-intercepts.
%
% Before you start to work predict what the graphs will look like * Is it going to open up
% or down? * Will the vertex be a maximum or a minimum? * Will it be wider or narrower than
% the parent? * Is the line of symmetry the y-axis, or is it shifted?
%
%Example 1
%
%Graph: y = 3x2
%
%Solution: * Should open up, have a minimum, be skinny, and an un-shifted line of symmetry
%* Axis of Symmetry: x = 0 * Coordinates of Vertex: y = 3(0)2 = 0 --> (0,0) * X- and
%Y-intercepts? --> it's vertex is the origin, so I've already found these * Find one other
%point\ldots{} let's plug in x = 1 --> y = 3(1)2 = 3, so plot (1,3). * Reflection across
%the axis of symmetry: (-1,3)
%
%
% Example 2 Graph Solution: * Should open down, have a maximum, be wide, and an un-shifted
% line of symmetry * Axis of Symmetry: x = 0 * Coordinates of Vertex: y = - ? (0)2 = 0 -->
% (0,0) * X- and Y-intercepts? --> it's vertex is the origin, so I've already found these *
% Find one other point\ldots{} let's plug in x = 4 --> y = (- ?) (4)2 = -4, so plot (4,-4).
% * Reflection across the axis of symmetry: (-4,-4)
%
%
%Example 3 Graph y = x2 + 5
%
%Solution: * Should open up, w/ minimum, same width as parent, and un-shifted line of
%symmetry * Axis of Symmetry: x = 0 * Coordinates of Vertex: y = (0)2 +5 = 5 --> (0,5) *
%Y-Intercept --> is the vertex * X-intercept --> replace y with zero and solve for x (by
%definition) --> oops, not real! * Find one other point\ldots{} let's plug in x = 1 --> y
%=(1)2 + 5= 6, so plot (1,6). * Reflection across the axis of symmetry: (-1,6)
%
%
%Example 4
%
%Graph y = 2x2 - 8
%
%Solution: * Should open up, w/minimum, skinny * Axis of Symmetry: x = 0 * Coordinates of
%Vertex: y = 2(0)2 -8 = 8 --> (0,8) * Y-Intercept --> is the vertex * X-intercept -->
%replace y w/0 and solve for x --> use your PoEs \& square rooting skills and you get +/- 2.
%(remember, when you choose to square root, you have to put the sign back on).
%
% Locating the Axis of Symmetry from the Standard Form Equation
%
%Now, what if I throw in that ``bx'', or, linear term? That will shift the axis of
%symmetry, but by how much? There is actually a formula based on the numbers in the
%standard form equation that will find the axis of symmetry.
%
%Equation for Axis of Symmetry of Parabola
%
%Given where a $ eq$0 the equation for the axis of symmetry is .
%
%Example 5
%
%Graph: y = 3x2 - 6x + 2
%
%Solution: 1. Open up, w/minimum, skinny, shifted line of symmetry 2. Axis of Symmetry: 3.
%Coordinates of Vertex: (1, -1)
%
%
%4. Y-Intercept is not the vertex this time --> (0,2) 5. X-intercept --> replace y with
%zero and solve for x (by definition) --> Uh, ewe. We can't find them in this equation.
%Every time we use a POE, we are left with something that we can't solve! Noooo! So, before
%we can plot these, we are going to have to learn a way to deal with that type of equation
%\ldots{} next class period! 6. We have to find other points. We have several options,
%depending on how accurate you want your graph to be * We can find the y-intercept's
%reflection and plot it --> (2,2) * We can plug in x = 3 to find an additional point -->
%use x = 3, and find (3,11) and (-1, 11) * We can do both of the above --> will give us the
%best graph
%
%Example 6
%
%Graph: y = x2 - 4x + 4
%
%Solution: 1. Open up, w/minimum, same width as parent, shifted line of symmetry 2. Axis of
%Symmetry: 3. Coordinates of Vertex: (2,0) --> oooo, the x-intercept!
%
%
%4. X-intercept --> we accidentally found that already 5. Y-Intercept is not the vertex
%this time --> (0,4) 6. We have to find other points. We have several options, depending on
%how accurate you want your graph to be * We can find the y-intercept's reflection and plot
%it --> (4,4)\ldots{} preferred * We can plug in x = 1 , 3, anything but 2 or 0 * We can do
%both of the above --> will give us the best graph
%
%Example 7
%
%Graph: Solution: 1. Opens up, w/minimum, wider, shifted line of symmetry 2.Axis of
%Symmetry: 3. Coordinates of Vertex: (-9,-36)
%
%4.Y-Intercept is not the vertex this time --> (0,-9)
%
%5. X-intercept --> nope, can't do that yet.. but I do know that there are 2\ldots{} it
%opens up and the vertex is below the x-axis 6. We have to find other points. We have
%several options, depending on how accurate you want your graph to be * We can find the
%y-intercept's reflection and plot it --> (-18,-9) * We can plug in x = anything but 9, 1'm
%going to pick 3 * We can do both of the above --> will give us the best graph Example 8
%
%Graph: Solution: 1. opens down, w/maximum, same width as parent, shifted line of symmetry
%2. Axis of Symmetry: 3. Coordinates of Vertex: (-2.5, -0.25)
%
%4. Y-Intercept is not the vertex this time --> (0,-6)
%
%5. X-intercept --> nope, can't do that yet.. but I do know that there are 2\ldots{} it
%opens down and the vertex is above the x-axis 6. We have to find other points. We have
%several options, depending on how accurate you want your graph to be * We can find the
%y-intercept's reflection and plot it --> (-5,-6) * We can plug in x = 3 to find an
%additional point --> use x = -2 or -3, and find \ldots{}oh, the x-intercepts!
%
%\section{The Vertex Form}
%
%Like linear functions, quadratic functions have more than one form for the equation. There
%are 3 that we look at in Algebra 1: 1. Standard --> the one we've been working with most
%2. Vertex --> the best for graphing 3. Factored --> the best for finding x-intercepts,
%which we'll look at last in the quadratic unit
%
%Vertex Form of a Quadratic: or where (h,k) are the coordinates of the vertex. Should look
%a little familiar \ldots{} point-slope anyone?
%
%This form is based on translating (shifting) the vertex of y = ax2 from (0, 0) to (h, k).
%* y = ax2 + k moves the vertex up or down k-units * y = a(x-h)2 moves the vertex left or
%right h units * The width and orientation of y = ax2 and should be the same.
%
%How to use the vertex form to graph.
%
%It gives you the coordinates of the vertex and therefore the location of the line of
%symmetry! That's half the work right there.
%
%Example 1
%
%Problem: What does y - 15 = 3 (x - 8)2 tell you about the graph.
%
%Solution: * The Vertex is located at (8, 15) * The line of symmetry is x = 8 * The 3 tells
%me it is thinner than the parent, opens up and has a minimum * The range of the function
%is y ? 15 * It has no x-intercepts
%
%The only thing I don't know, the location of the y-intercept. But, I can calculate that
%easily by substituting in x = 0 and solving for y.
%
%y - 15 = 3(0-8)2 y = 3(64) + 15 = 207
%
%Unpleasant to graph because of the horrible y-scale, but much easier to find the needed
%information for a graph.
%
%Converting from vertex to standard
%
%Not that you really want to but\ldots{}. Super easy. Just simplify! Don't forget Leo
%B./FOIL Example 2
%
%Problem: Convert
%
%Solution:
%
%
%
%
%Converting from standard to vertex\ldots{}.
%
%Something you really want to be able to do\ldots{} but can't yet. You need to learn a
%process called ``completing the square.'' Before I can teach you that, you have to learn
%about these things called polynomials. Then you have to learn how to factor a polynomial.
%Then you have to learn about these things called perfect square trinomials\ldots{} then
%you can learn how to complete the square. It is one of the last lessons of the year.
%
%
%\section{Solving Quadratics}
%
%Vocabulary
%
%Standard Form Quadratic Equation An equation of the form 0 = ax2 + bx + c, where a $ eq$
%0. It has a corresponding function y = ax2 + bx + c. Solving this equation will locate the
%x-intercepts of the corresponding function. Standard form quadratic equations must = 0!
%
%Vocabulary
%
%Quadratic Formula A formula that uses the coefficients of the terms of a standard form
%quadratic equation to solve for x. When 0 = ax2 + bx + c
%
%Vocabulary
%
%Discriminant The value under the radical of the quadratic formula. It is used to determine
%the number and type of solutions a quadratic equation will have. When 0 = ax2 + bx + c,
%the discriminant is b2 - 4ac.
%
%Recap from last time and some general solving information\ldots{}
%
%So last class, we graphed quadratics by hand. We know how to locate the line of symmetry,
%and the vertex. We can find the y-intercept in the equation. We can find x-intercepts if
%the b = 0, but we couldn't find them when there was a ``bx'' term. Our trusty ``POE's''
%failed! So, we need another way to solve equations that aren't linear. There are really 5
%ways to solve a quadratic equation.
%
%1. Graphing --> Only if you have a graphing calculator, and usually doesn't yield the
%exact answer. Just one that is an approximation
%
%2. Opposite Operations --> Gives exact answer, aka ``using the POEs'' only can be used if
%there is no ``bx'' term
%
%3. Factoring --> Easy, gives exact answers, doesn't work on every quadratic, but can be
%used on higher order equations. --> Learn this next six weeks
%
%4. Completing the square --> Easy, always works, based on factoring, works on any equation
%that can be written in quadratic form, but you need to know how to factor --> Learn in
%Algebra II
%
%5. Quadratic Formula --> It's a formula. Derived from completing the square. Difficulty
%only comes from simplification + there is a song to help you remember it!
%
%
%Methods \#1 and \#2
%
%Solving a quadratic with graphing
%
%1. Make it standard form\ldots{}''one side = 0'' 2. Graph y = standard form\ldots{} this
%is called the ``corresponding function'' to an equation. This works with all types of
%equations, not just quadratic 3. Look for x-intercepts 4. It has to = 0, or the whole
%``look for the x-intercepts'' thing isn't going to work. 5. x-intercepts are the
%roots/zeroes/solution to the equation.
%
%It is important to note that finding the ``roots'' or ``zeroes'' or ``solving'' or
%``x-intercepts'' are all basically the same thing. The context will dictate which word
%will be used.
%
%Solving a quadratic with opposite operations (POE's) --> only with special cases
%
%Example 1
%
%Problem: 5x2 - 20 = 0
%
%Solution: 5x2 = 20 --> APOE to move 20 over x2 = 4 --> DPOE to get x2 by itself x = ?2 -->
%Square root both sides, remembering to add the +/-
%
%Example 2
%
%Problem: 10 = 2x2 + 60
%
%Solution: -50 = 2x2 --> move variables to one side, numbers to the other with DPOE -25 =
%x2 --> Divide both sides by 2 No Real Solution --> can't take the square root of a
%negative number
%
%Example 3
%
%Problem: x2 + 12 = 4x2 - 12
%
%Solution: 24 = 3x2 --> move variables to one side, numbers to the other with DPOE 8 = x2
%--> Divide both sides by 2 --> square root both sides, remember the +/- --> Simplify the
%radical!
%
%Example 4
%
%Problem: (x -4)2 + 8 = 24
%
%
%Solution: (x-4)2+8 = 24 (x-4)2 = 16 -->isolate the parenthesis --> square root both sides,
%don't forget the +/- x - 4 = ?4 --> square root both sides. Be sure to simplify the
%radical x = 4 ? 4 --> add 4 to isolate the x x = 8 and x = 0 --> split into 2 answers.
%
%
%
%Example 5
%
%Problem: -2(x+8)2 - 12 = 36
%
%Solution: -2(x+8)2 = 48 --> add 12 to both sides (x+8)2 = -24 --> divide both sides by -2
%to isolate the ( )
%
%Notice that something that is squared is equal to something negative, therefore S = ?
%
%Example 6
%
%Problem: 2(x - 5)2 + 12 = 52
%
%
%Solution: 2(x - 5)2 = 40 --> subtract 12 from both sides (x - 5)2 = 20 --> divide both
%sides by 2 --> square root both sides --> simplify the radical --> add 5 to both sides -->
%split into two solutions
%
%Methods \#5\ldots{}
%
%The formula\ldots{} comes from factoring and completing the square. ALWAYS WORKS, ALWAYS
%GIVES THE EXACT ANSWER\ldots{} you can approximate if necessary. The Standard Form of a
%Quadratic EQUATION (not function) is ax2 + bx + c = 0. It has to be zero! It is based on
%where the x-intercepts are in relation to the line of symmetry for the corresponding
%function.
%
%In order to use the formula, you must have the equation in standard form. That means, one
%side of the equation has to be zero. I don't know how many times I am going to say
%this\ldots{} it is really important!!!!!
%
%So given ax2 + bx + c = 0, the value of x can be determined by
%
% when simplifying you can split it up to , notice \ldots{}the values of x come from
% \ldots{} line of symmetry + something, line of symmetry - something
%
%
%Steps to Solving\ldots{}
%
%1. Make sure the equation is in standard form 2. State what a, b, and c are 3. Plug into
%formula 4. Simplify (where it gets tricky) 5. Write answer in asked for form 6. Check
%
%
%Example 7
%
%Solve: Solve x2 - 5x + 6 = 0
%
%Solution:
%
%
%It is in the correct form. a = 1 b = -5 c = 6 plug into the formula and simplify
%
%Now, split it into the 2 answers and keep simplifying.
%
%Write your solution in set notation.
%
%Example 8
%
%Solve: Solve 4x2 +4x = -1
%
%Solution:
%
%
%It is not in the correct form. Move the -1 over
%
%a = 4 b = 4 c = 1 plug into the formula and simplify x = -4/8 = - ?
%
%S = {- ? } No need to split here\ldots{} the radical was zero. Which means that there is
%one solution (vertex is the x-intercept) Write your solution in set notation.
%
%
%
%Example 9
%
%Solve: 3x2 - 6x + 2 = 0
%
%Solution:
%
%
%It is in the correct form.
%
%a = 3 b = -6 c = 2 plug into the formula and simplify
%
%Now Split, and simplify each part.
%
%Write your solution in set notation.
%
%If the denominators are the same, you can squish it back together.
%
%Now, if you need a decimal approximation, find it using these values.
%
%Example 10
%
%Solve: x2 + 3x = -6
%
%Solution:
%
%
%It is not in the correct form. Move the -6 over
%
%a = 1 b = 3 c = 6 plug into the formula and simplify Can you go any further? Nope, as soon
%as you see that negative value under the radical, you can start. This means that there is
%no real solution to this quadratic. Which means the corresponding function will have no
%x-intercept. Now, an easy check. if you suspect no solution, graph to make sure. It is so
%easy to make a sign mistake!
%
%How many solutions does a quadratic have\ldots{} from a, b, and c?
%
%Now, in the previous example, you should notice that the value under the radical tells you
%how many solutions there will be. This value, b2 - 4ac has a name. It is called the
%discriminant.
%
%If the discriminant is negative = no real solution, because you can't square root a
%negative number If the discriminant is zero = 1 real solution, because you aren't adding
%or subtracting anything If the discriminant is positive = 2 real solutions. (These have to
%be a perfect square in order for the solutions to be a rational number. Otherwise you get
%a radical left over\ldots{})
%
%So before you solve, you can determine the number of solutions, and maybe save yourself
%some work. You would do this after step 2 in ``steps to solving a quadratic''
%
%Example 11
%
%Solve:
%
%3x2 + 9x + 18 = 0 Solution: Find the discriminant a = 3, b = 9 , c = 18
%
%(9)2 - 4(3)(18) = 81 - 216 = -135\ldots{} the discriminant is negative, no solution! Done!
%
%
%\section{Projectile Motion}
%
%Vocabulary
%
%Projectile Any object that moves through the air or through space acted on only by the
%force of gravity.
%
%Projectiles
%
%* A thrown baseball * An arrow that has been shot from a bow.
%
%Not Projectiles
%
%* Anything that uses energy to stay in the air: airplanes, birds, etc. * Anything that is
%dramatically affected by air resistance.
%
%Vocabulary
%
%Trajectory An imaginary tracing of a projectile's positions as it moves through space.
%
%Trajectory of a projectile
%
%* The trajectory of a projectile is a parabola. * The motion of a projectile can be
%explained using a quadratic equation.
%
%Falling Objects
%
%* An object that has been dropped and falls through space is one example of a projectile.
%* Gravity causes the object to accelerate towards the ground.
%
%Gravity
%
%* Causes falling objects to accelerate (increase in speed) as time passes. * The
%Acceleration of Gravity o -32 feet per second per second or ft/s2 o -9.8 meters per second
%per second or m/s2
%
%It is a negative quantity because acceleration is a value with a direction. It will slow
%down an object that is thrown upwards, and pulls dropped objects towards the ground. For
%our equations for falling objects, and vertical launch, you will use the negative value.
%Remember acceleration is the change in velocity over time. That is why the units are
%``feet per second, per second''
%
%This graph shows the height of an object that has been dropped over time. Notice how the
%curve of the graph gets steeper the longer it is falling.
%
%
%h = the height of the projectile above the ground a = the acceleration of gravity t = the
%time (in seconds) h0 = the starting height
%
%This equation is used to find the height (at time t) of an object that is falling towards
%the ground. Be careful when applying this formula. You must use the value of ``a'' that
%will match with the unit you are given for height.
%
%
%Example 1
%
%Problem: Mr. Campbell drops a can of SPAM off a bridge that is 80 feet high. When will the
%SPAM hit the ground?
%
%
%
%Solution: When an object hits the ground, h is usually 0. So, h = ? at2 + ho 0 = ? (-32)t2
%+ 80 --> solve for t. -80 = -16t2 5 = t2 2.24 sec
%
%Technically, when you square root both sides, you get a positive and a negative quantity,
%but the negative solution doesn't make sense in the context of the problem.
%
%Horizontal Launch
%
%Objects that are thrown or launched horizontally travel in two dimensions - both
%horizontally and vertically. However, the vertical and horizontal motion are independent.
%To describe the vertical motion, use the equation for a falling object. To describe the
%horizontal motion, use the equation d = r?t where r is the object's initial horizontal
%velocity!
%
%If you need to find out how long it will take a horizontally launched object to hit the
%ground, treat it like a falling object. The only thing you need to remember about
%horizontally launched objects is that they will land further away from the launch site
%than an object that is dropped.
%
%Example 2a Problem: Mr. Campbell fires a can of SPAM horizontally off the edge of an
%80-meter high cliff. When will the SPAM hit the ground? Solution: When it is hitting the
%ground = solve for t in the falling object equation. h = ? at2 + ho 0 = ? (-9.8)t2 + 80
%--> meters so use -9.8 for the acceleration of gravity 0 = -4.9t2+ 80 -80 = -4.9t2
%16.32653061 = t2 4.04 sec = t.
%
%Example 2b Problem: If the initial velocity of the SPAM was 120 m/s, how far from the base
%of the cliff will be the point of impact?
%
%Solution: The example before this one gives you flight time. Now that we know how long the
%object will be in the air, we can figure out how far from the base of the cliff the object
%should land. D = rt --> D = 120(4.04) --> D = 484 meters!
%
%Vertical Launch
%
%* Objects that are thrown or launched vertically travel in only one dimension. * The
%rising and falling components of their motion are symmetrical.
%
%These objects do not include things like the space shuttle. That is not a projectile! It
%uses additional energy to travel upwards. However, the soda bottle rockets that run on an
%initial burst of compressed air are vertically launched projectiles.
%
%This equation is just like the falling object equation, but it has the ``vt'' added to it.
%This represents the distance traveled because of the upward motion from the launch.
%
%
%
%h = the height of the projectile above the ground at time t a = the acceleration of
%gravity t = the time (in seconds) v = the initial vertical velocity h0 = the starting
%height
%
%
%Example 3a
%
%Problem: Mr. Campbell fires a can of SPAM vertically from the ground with an initial
%velocity of 96 ft/s. When will the SPAM reach its maximum height?
%
%Solution: If you need to find the time it reaches its maximum height, think about the
%shape of the graph. It is a parabola. Where does the maximum occur? It occurs at the line
%of symmetry. h = ? at2 + vt + ho h = ? (-32)t2 + 96t + 0 h = -16t2 + 96t + 0
%
%The coefficient of the t2 term is ``a'' and the coefficient of the t term is ``b''
%
%t = -b/2a --> equation to find the line of symmetry, or the time when the projectile will
%reach its maximum height t = -96 / 2(-16) = 3 seconds.
%
%Example 3b
%
%Problem: What is the maximum height of the can of SPAM?
%
%Solution: So, now we are looking for the maximum height\ldots{} what is that but the
%y-value of the vertex. Substitute in 3 for t in the equation. h = ? at2 + vt + ho h = ?
%(-32)t2 + 96t + 0 h = ? (-32)(3)2 + 96(3) = 144 ft
%
%
%Example 3c
%
%Problem:
%
%How long will it take the can of SPAM to fall from this height? What is its total flight
%time?
%
%Solution:
%
%Well, if the object is launched from the ground, the amount of time it takes to reach the
%maximum height is the same amount of time it takes to hit the ground\ldots{}. So 3
%seconds, which gives a total flight time of 6 seconds.
%
%Graph of vertically launched object's height above the ground over time when the object
%wasn't launched from the ground .
%
%X-intercept is where it hits the ground.
%
%
%Example 4a
%
%Problem: Mr. Campbell fires a can of SPAM vertically from the top of the 80-meter
%(240-foot) cliff with initial velocity of 128 ft/s. When will the SPAM reach its maximum
%height?
%
%Solution: Now when the object is not launched from the ground, you can do the same thing
%to find the maximum height\ldots{} just find the axis of symmetry. h = ? (-32)t2 + 128t +
%240 h = -16t2 + 128t +240 t= -128/2(-16) = -128/-32 = 4 seconds
%
%
%
%Example 4b
%
%Problem: What is the maximum height of the can of SPAM?
%
%Solution: Now, substitute in 4 seconds into the equation to find the maximum height. h = ?
%(-32)t2 + 128t + 240 h = -16t2 + 128t +240 h = -16(4)2 + 128(4) +240 = 496 ft.
%
%
%Example 4c
%
%Problem: What is its total flight time?
%
%Solution: Now, for this one, you can't just say it is 8 seconds. The object didn't start
%on the ground and the line of symmetry is not the y-axis. In order to figure this out, we
%are going to have to solve for t when h is 0. One solution will be negative, and that just
%doesn't make sense to the context. h = -16t2 + 128t +240 0 = -16t2 + 128t +240
%
%when it hits the ground will be the total flight time. You must use the quadratic formula
%
%
% So, = 9.5678 seconds.
%
%\section{Quadratics From Data}
%
%Under construction.