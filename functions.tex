\chapter{Fundamentals of Functions}
\label{ch:functions}
 
\section{A Modest Proposal}

I'm having a hard getting this chapter to flow. I've got over it and over it, and I just haven't been able to take the version 1 material and get it to work\ldots\ whereas the other parts of the book flow so well.

I've been thinking about our talk the other week about the course coming in waves, and how this chapter is kind of the crest of the wave that began in chapter 1, and that after this point we start the new linear wave.

I've thought long and hard and I have a modest proposal.

I say we cut this chapter, redistribute the concepts to other parts of the book, and replace it with a ``capstone section'', consisting only of the ``exploring the three families'' activity.

In the capstone section, we use the activity to preview transformations, and talk about things informally. But, we don't get into any kind of $f(x)+c$ detail until we get into each of the individual families. For example, the transformations talk would first appear after we've done direct variation, and gotten into the linear forms. We focus at that time on transformations only of linear functions. Transformations of exponentials happens in the exponential unit.

Note that ``big graphs'' is still there in the previous chapter, and so we'd still have informal language available about shifting up/down, steeper/shallower, wider/narrower (between big graphs and exploring the families).

We save domain and range discussions until we need them. Linear functions have nothing interesting to say about domain/range, so save it. When have some more experience about how exponential equations work, the infinite progression of fractions leading to an asymptote, and how that means that there's a distinction between output values of linear and exponential functions --- then we discuss domain and range.

Finally, all the ``is it a function'' stuff feels like a kind of distraction. Why is the concept so important at this very moment? I like the idea of calling things by their proper names, and if we're going to study ``functions'' we should define that term. But, I don't think we need all the detail.

So, here's what I picture: A short introductory bit that defines function, gives some examples and nonexamples, but basically says: We're going to study functions in algebra 1. There are things that aren't functions, but we'll learn about those things later.

I say we introduce function notation, since for the most part it's just a renaming. Plus, it gives a handy notation for evaluation.

Then, I say we do the activity ``exploring the families'' (with the sorting of equations, graphs, and informal discussions of transformations) and explain the key features of each family (table, equation graph).

And then we start the linear unit.

Thoughts?

%%\begin{quote}
%%%A mathematician is a machine for turning coffee into theorems.
%%%\par\hfill --- Paul Erd\H{o}s, Hungarian mathematician
%%On two occasions I have been asked, ``Pray, Mr. Babbage, if you put into the machine wrong figures, will the right answers come out?'' I am not able rightly to apprehend the kind of confusion of ideas that could provoke such a question.
%%\par\hfill --- Charles Babbage, English inventor of the first mechanical computer
%%\end{quote}
%%
%%% % % % % % % % % % % % % % % % % % % % % % % % % % % % % % % % % % % % % % % % 
%%\section{Mathematical Relationships}
%%\label{sec:mathrelationships}
%%
%%%\begin{boxedexplore}[Name of Extended Exploration]
%%%\addtodoitem{Click here to visit the extended exploration: NAME}
%%%\end{boxedexplore}
%%
%%\begin{boxedexplore}[Startup Exploration: Number Machines]
%%A number machine accepts numbers as input and produces numbers as output. When machine A receives the number 8 as input, it produces the output 28. When this machine receives $\umin12$ as input, the output is $\umin42$.
%%
%%Machine B has a different mechanism for producing output values. When machine B receives the number 8 as input, it outputs 15. When it receives $\umin12$ as input, the output is $35$.
%%
%%What do you suppose each machine will output when given the number $\umin7$ as input? Write a sentence explaining how you believe each machine works.
%%
%%Can you find alternative explanations for how each machine works? Be creative!
%%\end{boxedexplore} %% End of startup exploration
%%
%%Relationships lie at the heart of algebra. We have seen a number of relationships so far, for example the relationship between distance and time of a moving object, or the relationship between a number's position in a sequence and that number's value. The number machines described in the startup exploration define a kind of relationship between input values and output values.
%%
%%\begin{boxeddef}[Relation]
%%A \gls{relation} defines how certain numbers (or other objects) are connected to other numbers (or objects). We can think of a relation as a collection of ordered pairs $(x,y)$, meaning ``$x$ is related to $y$''.
%%\end{boxeddef}
%%
%%Not all relationships are created equal, however. In \cref{sec:interpgraphs} we saw that certain graphs could be drawn in a way that would suggest an impossible situation. For example, we drew a graph (shown in \cref{fig:impossible}) that suggested Yeardleigh's submersible could at three different depths simultaneously, which is nonsense.\footnote{``I like nonsense; it wakes up the brain cells.'' --- Dr. Seuss}
%%
%%\begin{figure}[!htbp]
%%\centering
%%\begin{tikzpicture}[scale=1]
%%	\draw[ultra thin, gray!50] (0,0) grid (6,5);
%%	\draw[->,thick] (0,0) -- (6.5,0);
%%	\draw[->,thick] (0,0) -- (0,5.5);
%%	%% x-axis
%%	\foreach \x in {0,...,6}
%%		\draw (\x,0.1) -- (\x,-0.1);
%%	\foreach \x in {0,15,...,90}
%%		\draw (0.0667*\x, 0) node[below] {\footnotesize\x};
%%	\draw (3.25, -0.75) node[below] {Elapsed Time (min)};
%%	%% y-axis
%%	\foreach \y in {0,...,5}
%%		\draw (-0.1,\y) -- (0.1,\y);
%%	\foreach \y in {0,100,...,500}
%%		\draw (0, 0.01*\y) node[left] {\footnotesize\y};
%%	\draw (-1, 2.5) node[rotate=90,above] {Depth (ft)};
%%	%% graph
%%	\draw[very thick, blue] (0,0) -- (3,1.5) -- (1,2.5) -- (4,4) -- (6,0);
%%	\draw[very thick, white] (2,0) -- (2,5);
%%	\draw[very thick, dotted, green!60!black] (2,0) -- (2,5);
%%	\draw[white,fill=green!60!black] (2,1) circle[radius=0.1];
%%	\draw[white,fill=green!60!black] (2,2) circle[radius=0.1];
%%	\draw[white,fill=green!60!black] (2,3) circle[radius=0.1];
%%\end{tikzpicture}
%%	\caption{A graph of depth over time? Impossible!}
%%	\label{fig:impossible}
%%\end{figure}
%%
%%Many of our other relationships would have a similar problem. It would be nonsense to suggest that a sequence had more than one value as its ``third term'', for example. Depending on how we think about ``number machines'', we might expect that a machine would act predictably and always produce the same output for a given input.
%%
%%Relationships that follow this very natural rule --- you can't be in more than one place at the same time --- have a special status in algebra. They're called \glspl{function}.
%%
%%%From a certain point of view, there is nothing wrong with this graph --- any old squiggly line in the coordinate plane is a kind of graph. The problem arises because this graph doesn't make any sense \textit{in the context of the problem}. In this chapter, we will make some formal definitions that will help us to distinguish certain types of mathematical rules and relationships from one another.
%%
%%\begin{boxeddef}[Function]
%%A \gls{function} is a special type of relation in which the ordered pairs have the following property: each $x$-value is paired with one, and only one, $y$-value.
%%\end{boxeddef}
%%
%%The ``one and only one'' requirement is what makes a function special. The graph in \cref{fig:impossible} violates this requirement: certain $x$-values have more than one associated $y$-value (look at $x$ values between 15 and 45 minutes). So, this graph does not depict a function. Any given sequence is a function, since each $x$ value (position in the sequence) is occupied by exactly one $y$ value.
%%
%%\subsubsection{Representing Relations}
%%
%%Ther are several ways to represent a mathematical relation. We might represent a relation using:
%%\begin{itemize}
%%\item A graph. This representation gives is a visual of the relationship between input values and output values.
%%\item A table of values. This representation gives us an organized list of input values and their corresponding output values. (Related forms include a ``mapping diagram'', or just a collection of ordered pairs.)
%%\item An equation. This representation gives us a rule for turning input values into output values, the way a number machine might do.
%%%-- A mapping diagram. Like a table, gives a series of input and corresponding output values.
%%%-- A set of points. Like a table, but organized differently
%%\end{itemize}
%%Since functions hold a special position, our first task is to figure out how to determine whether a given mathematical relationship is, in fact, a function. To do this we have to check it against the definition of what it means to be a function. In other words, we must make sure each $x$-value corresponds to one and only one $y$-value. There are different features we can look for depending on the representation of the function, but the key question is always the same: does every $x$ correspond to a unique $y$?
%%
%%\subsection{Graphs of Functions}
%%
%%In order for a graph to be a function it must avoid the situation in which multiple $y$-values stack up above a particular $x$-value. A handy way of remembering this requirement is that the graph of a function passes the \gls{vertical line test}. The vertical line test states that for the graph of a function, any vertical line drawn on the graph will intersect the graph \textit{exactly once}.
%%
%%\begin{boxedex}
%%Which of the graphs below, if any, represents a function? How do you know?
%%
%%% Graph 1
%%\begin{minipage}{0.33\textwidth}
%%\centering
%%Graph A
%%\par\begin{tikzpicture}[scale=0.55]
%%	\draw[ultra thin, gray!50] (-4,-4) grid (4,4);
%%	\draw[<->,thick] (-4,0) -- (4,0);
%%	\draw[<->,thick] (0,-4) -- (0,4);
%%%	\draw (0, 4.75) node {Graph A};
%%	%% graph
%%	\draw[ultra thick, red, domain=-4:4] plot (\x, {3*sin(\x r)});
%%\end{tikzpicture}
%%\end{minipage}
%%% Graph 2
%%\begin{minipage}{0.33\textwidth}
%%\centering
%%Graph B
%%\par\begin{tikzpicture}[scale=0.55]
%%	\draw[ultra thin, gray!50] (-4,-4) grid (4,4);
%%	\draw[<->,thick] (-4,0) -- (4,0);
%%	\draw[<->,thick] (0,-4) -- (0,4);
%%%	\draw (0, 4.75) node {Graph B};
%%	%% graph
%%	\draw[ultra thick, violet, domain=-3:3] plot (\x,0.667*\x+2);
%%	\draw[ultra thick, violet, domain=-3:3] plot (\x,-0.667*\x-2);
%%\end{tikzpicture}
%%\end{minipage}
%%% Graph 3
%%\begin{minipage}{0.33\textwidth}
%%\centering
%%Graph C
%%\par\begin{tikzpicture}[scale=0.55]
%%	\draw[ultra thin, gray!50] (-4,-4) grid (4,4);
%%	\draw[<->,thick] (-4,0) -- (4,0);
%%	\draw[<->,thick] (0,-4) -- (0,4);
%%%	\draw (0, 4.75) node {Graph C};
%%	%% graph
%%	%\draw[ultra thick, blue] (0,0) circle[radius=3cm];
%%	\draw[ultra thick, blue, scale=1.27,domain=-3.141:3.141,smooth,variable=\t] plot ({\t*cos(\t r)},{\t*sin(\t r)});
%%\end{tikzpicture}
%%\end{minipage}
%%
%%\exsoln\ Only Graph A represents a function. We can draw vertical lines on Graphs B and C which intersect the graph at more than one point. This means that those $x$-values have more than one $y$-value, violating the definition of function.
%%\end{boxedex}
%%
%%Note that in Graph C, certain sections of the graph pass the vertical line test. There is no partial credit, though. A graph fails the vertical line test if any vertical line (even just one) crosses the graph in more than one point. This happens also with the graph in \cref{fig:impossible}. That graph fails the vertical line test for some vertical lines. Therefore the relationship depiected is not a function.
%%
%%\begin{boxedex}
%%Which of the scatter plots below, if any, represents a function? How do you know? You may have to look closely!
%%
%%% Graph 1
%%\begin{minipage}{0.33\textwidth}
%%\centering
%%Plot A
%%\par\begin{tikzpicture}[scale=0.55]
%%	\draw[ultra thin, gray!50] (-4,-4) grid (4,4);
%%	\draw[<->,thick] (-4,0) -- (4,0);
%%	\draw[<->,thick] (0,-4) -- (0,4);
%%%	\draw (0, 4.75) node {Plot A};
%%	%% graph
%%	\draw[blue] plot[only marks, mark=*, mark size=4] coordinates {(-4,3) (-3,0) (-2,1) (-1,-3) (0,4) (1,-2) (2,-1) (3,-1) (4,2)};
%%\end{tikzpicture}
%%\end{minipage}
%%% Graph 2
%%\begin{minipage}{0.33\textwidth}
%%\centering
%%Plot B
%%\par\begin{tikzpicture}[scale=0.55]
%%	\draw[ultra thin, gray!50] (-4,-4) grid (4,4);
%%	\draw[<->,thick] (-4,0) -- (4,0);
%%	\draw[<->,thick] (0,-4) -- (0,4);
%%%	\draw (0, 4.75) node {Plot B};
%%	%% graph
%%	\draw[red] plot[only marks, mark=*, mark size=4] coordinates {(-4,4) (-3,1) (-2,-3) (-1,1) (0,0) (1,-1) (2,2) (3,1) (3,-2) (4,3)};
%%\end{tikzpicture}
%%\end{minipage}
%%% Graph 3
%%\begin{minipage}{0.33\textwidth}
%%\centering
%%Plot C
%%\par\begin{tikzpicture}[scale=0.55]
%%	\draw[ultra thin, gray!50] (-4,-4) grid (4,4);
%%	\draw[<->,thick] (-4,0) -- (4,0);
%%	\draw[<->,thick] (0,-4) -- (0,4);
%%	%% graph
%%	\draw[green!80!black] plot[only marks, mark=*, mark size=4] coordinates {(-4,1) (-3,1) (-2,1) (-1,1) (0,1) (1,1) (2,1) (3,1) (4,1)};
%%\end{tikzpicture}
%%\end{minipage}
%%
%%\exsoln\ Plots A and C both pass the vertical line test, and therefore represent functions. Plot B fails the vertical line test because it fails for the vertical line at $x=3$.
%%\end{boxedex}
%%
%%Note that we don't care if different $x$'s map to the same $y$. For example, in Plot C, all of the $x$-values map to the same $y$-value. That's OK. In other words, a function \textit{can} fail the ``horizontal line test'' without penalty.\footnote{Later, in algebra 2, we'll discuss in more detail what it means to pass or fail the horizontal line test.}
%%
%%\subsection{Functions From Points}
%%
%%A scatter plot gives a visual representation of a collection of ordered pairs. If the ordered pairs are presented in an alternative format --- say, a table of values --- we can use the same reasoning to determine whether the given relation is a function.
%%
%%\begin{boxedex}
%%Which of the data sets below, if any, represents a function? How do you know?
%%
%%\begin{minipage}[t]{0.33\textwidth}
%%\centering
%%Set A
%%\par\begin{tabular}{|C{1.5cm}|C{1.5cm}|}
%%\hline
%%x & y \\\hline
%%-3 & 4\\
%%-2 & 6\\
%%-1 & 4\\
%%0 & 2\\
%%1 & 0\\
%%2 & -2\\
%%3 & 0\\\hline
%%\end{tabular}
%%\end{minipage}
%%% Graph 2
%%\begin{minipage}[t]{0.33\textwidth}
%%\centering
%%Set B
%%\par\begin{tabular}{|C{1.5cm}|C{1.5cm}|}
%%\hline
%%x & y \\\hline
%%2 & 1\\
%%1 & 2\\
%%-1 & 7\\
%%4 & 3\\
%%3 & 4\\
%%1 & 0\\
%%0 & 6\\\hline
%%\end{tabular}
%%\end{minipage}
%%% Graph 3
%%\begin{minipage}[t]{0.33\textwidth}
%%\centering
%%Set C
%%\par$(0,4) \quad (-1,6)$
%%\par$(2,3) \quad (0,4)$
%%\par$(1,5) \quad (-2,5)$
%%\end{minipage}
%%
%%\exsoln\ Set A is a function. There are no repeated $x$-values! Set B is not a function. The $x$-value 1 appears twice and with two different related $y$-values (0 and 2). This violates the definition of function. Set C is a function. The $x$ value 0 appears twice, but it is mapped to the same value in both cases.
%%\end{boxedex}
%%
%%A \gls{mapping diagram} is a similar way of representing a relationship between specific input and output values. Here we use two ovals and arrows. In one oval, we list the input values, in the other oval we list the output values. Then we draw arrows to show which $x$ maps to which $y$.
%%
%%\begin{boxedex}
%%Which of the mapping diagrams below, if any, represents a function? How do you know?
%%
%%\begin{minipage}{0.5\textwidth}
%%\centering
%%Diagram A
%%\par\begin{tikzpicture}[scale=0.55]
%%	\draw (0,0) circle[x radius=1.5, y radius=3];
%%	\draw (5,0) circle[x radius=1.5, y radius=3];
%%	\draw (0, 1.5) node[left]{1};
%%	\draw (0, 0.5) node[left]{2};
%%	\draw (0, -0.5) node[left]{3};
%%	\draw (0, -1.5) node[left]{4};
%%	\draw (5, 1.5) node[right]{2};
%%	\draw (5, 0.5) node[right]{6};
%%	\draw (5, -0.5) node[right]{9};
%%	\draw (5, -1.5) node[right]{0};
%%	\draw[thick,->] (0,1.5) -- (5,1.5);
%%	\draw[thick,->] (0,0.5) -- (5,-1.5);
%%	\draw[thick,->] (0,-0.5) -- (5,0.5);
%%	\draw[thick,->] (0,-1.5) -- (5,-0.5);
%%\end{tikzpicture}
%%\end{minipage}
%%% Graph 2
%%\begin{minipage}{0.5\textwidth}
%%\centering
%%Diagram B
%%\par\begin{tikzpicture}[scale=0.55]
%%	\draw (0,0) circle[x radius=1.5, y radius=3];
%%	\draw (5,0) circle[x radius=1.5, y radius=3];
%%	\draw (0, 1.5) node[left]{1};
%%	\draw (0, 0.5) node[left]{2};
%%	\draw (0, -0.5) node[left]{3};
%%	\draw (0, -1.5) node[left]{4};
%%	\draw (5, 1) node[right]{9};
%%	\draw (5, -1) node[right]{7};
%%	\draw[thick,->] (0,1.5) -- (5,-0.8);
%%	\draw[thick,->] (0,0.5) -- (5,1.1);
%%	\draw[thick,->] (0,0.5) -- (5,-1);
%%	\draw[thick,->] (0,-0.5) -- (5,0.9);
%%	\draw[thick,->] (0,-1.5) -- (5,-1.2);
%%\end{tikzpicture}
%%\end{minipage}
%%
%%\exsoln\ The first mapping is a function. Each $x$-value maps to exactly one $y$-value. The second mapping is not a function: the value 2 maps to both 7 and 9.
%%\end{boxedex}
%%
%%Note again that the problem in Diagram 2 is \textit{not} that multiple arrows point to each of the output numbers. Multiple ``in'' arrows is fine. The problem is that 2 has multiple ``out'' arrows that go to different targets.
%%
%%%The table, the mapping diagram, and the list of points are all basically the same sort of representation: we are presented with a finite number of ordered pairs. So, these representations only give a limited picture of how the function operates. The preferred representations are the graph and the equation.
%%
%%% If you are given one of the others, you will likely have to convert it to a graph or an equation.
%%
%%\subsection{Equations of Functions}
%%
%%The ``number machine'' metaphor is often helpful when thinking about a function. A number (an $x$-value) goes into the machine, the machine's rule works on the number, and another number comes out of the machine (the $y$-value). Most of the time, when we have a particular rule and a particular input, we get a single, predictable output.
%%
%%For example, the equation $y = 3x$ defines a function. We can substitute in 5 for $x$. The function machine multiplies 5 by 3 and outputs 15. This generates the ordered pair $(5, 15)$ and we say that ``5 maps to 15''. If we choose other values of $x$, we can generate more ordered pairs.
%%
%%Not every equation defines a function, though. In order to be a function, every input value must produce one and only one output value. In algebra 2 we will study many interesting and important mathematical relationships that are not functions. We will learn, for instance, how to write the equation for a circle --- and a circle is not a function. (Can you explain why not?)
%%
%%Our focus in algebra 1 will be on three main ``families'' of functions, and our rules will all be of the form ``$y$ equals some expression in terms of $x$''.
%%
%%%In addition to being able to determine if a relation is a function, one has to be able to convert between the different representations. We will spend more time on this when we study each family of functions in depth.
%%
%%\subsubsection{Function Notation}
%%
%%%\begin{boxedexplore}[Name of Extended Exploration]
%%%\addtodoitem{Click here to visit the extended exploration: NAME}
%%%\end{boxedexplore}
%%%
%%%\begin{boxedexplore}[Name of Startup Exploration]
%%%Description of startup exploration.
%%%\end{boxedexplore} %% End of startup exploration
%%
%%We can thank Swiss mathematician Leonhard Euler for the concept of a function.\footnote{The surname Euler is German, and so it is pronounced $OIL \cdot er$ (and \textit{not} $YOO \cdot ler$).} He was the first to coin the term ``function'' and was the first to use a handy way of writing a function, ``function notation''.
%%
%%Up until now, we have used ``$y = $ something in terms of $x$'' to define every function. Euler's function notation looks a bit different, but it is quite helpful in certain situations. The generic form of function notation is ``$f(x)~=~$something in terms of $x$''. In this case, we simply replace $y$ with the notation $f(x)$, which is read aloud as ``$f$ of $x$'' or ``$f$ as a function of $x$''.
%%
%%In its generic form, the letter $f$ is used to name the function. We use the letter $f$, of course, because it stands for ``function''. And as usual, we use $x$ to represent the independent variable.
%%
%%What's nice about this notation is that it allows us great flexibility to use other variables that can be more descriptive. For example, if we wanted to describe how the height of a bouncing ball changes over time, we might want to let the variable $t$ represent time, and then name the function with $h$ for height.
%%
%%So, we could write a function ``$h(t) = $ some expression in terms of $t$''. That's read ``$h$ of $t$'', which does a pretty good job of capturing the idea that we're describing \textit{height} in terms \textit{of time}.
%%
%%We might write a function like $d(t) = 60t$ which described distance $d$ as a function of time $t$. This is exactly the same equation as $y = 60x$, but we've changed the names of things to better match the context.
%%
%%\subsubsection{Evaluation Using Function Notation}
%%
%%Function notation has another convenience. We can use the notation to indicate a specific choice of value for the independent variable. Earlier, we said ``evaluate the function $y = x^2 - 4$ when $x = 3$''. This was fine, but function notation gives us a way to shorten these instructions.
%%
%%For example, if we have $f(x) = x^2 - 4$ then we might want to evaluate $f(3)$, which is said aloud ``$f$ of three''. The $x$ in the function notation was replaced by 3, which is exactly what is means to evaluate the function at that value! The notation is telling us to input 3 for $x$ and find the output value. So if $f(x) = x^2 - 4$, then $f(3) = (3)^2 - 4 = 9 - 4 = 5$.
%%
%%We will use function notation and $y=$ notation through this course. The flexibility of function notation means that it is the preferred way of writing functions in higher level mathematics courses.
%%
%%\section{More to come in the functions chapter}
%%



%% % % % % % % % % % % % % % % % % % % % % % % % % % % % % % % % % % % % % % % % 
%\section{The Three Families of Algebra 1}
%
%The majority of algebra 1 will focus on the in-depth study of three key families of functions: the linear family, the exponential family, and the quadratic family. We've encountered each of these families before, but now it's time for a formal introduction.
%
%For a given function to be considered a member of one of these families, it must share the characteristics of other members of that family as seen in its graph, the patterns in its values, and in its equation.
%
%In the previous section we learned techniques for identifying whether a given relation is a function. One of our tasks going forward will be to learn techniques for distinguishing these three function families from one another.
%
%\begin{boxedexplore}[Exploring the Three Families]
%\addtodoitem{Click here to visit the extended exploration: Exploring the Three Families}
%\end{boxedexplore}
%
%\subsection{Families and Parents}
%
%\begin{boxedexplore}[Startup Exploration: Acting Like Your Parents]
%
%We call the most basic member of the family the \gls{parent function} for the family. The rules and graphs of the parent functions for each of the three families are shown below.
%
%% Graph 1
%\begin{minipage}{0.33\textwidth}
%\centering
%Linear Family
%\\ $y=x$
%\par\begin{tikzpicture}[scale=0.45]
%	\draw[ultra thin, gray!50] (-5,-5) grid (5,5);
%	\draw[<->,thick] (-5,0) -- (5,0);
%	\draw[<->,thick] (0,-5) -- (0,5);
%	\draw[<->, ultra thick, blue, domain=-5:5] plot (\x,\x);
%\end{tikzpicture}
%\end{minipage}
%% Graph 2
%\begin{minipage}{0.33\textwidth}
%\centering
%Exponential Family
%\\ $y = 2^x$
%\par\begin{tikzpicture}[scale=0.45]
%	\draw[ultra thin, gray!50] (-5,-1) grid (5,9);
%	\draw[<->,thick] (-5,0) -- (5,0);
%	\draw[<->,thick] (0,-1) -- (0,9);
%	\draw[<->, ultra thick, blue, domain=-5:3.15] plot (\x,2^\x);
%\end{tikzpicture}
%\end{minipage}
%% Graph 3
%\begin{minipage}{0.33\textwidth}
%\centering
%Quadratic Family
%\\ $y = x^2$
%\par\begin{tikzpicture}[scale=0.45]
%	\draw[ultra thin, gray!50] (-5,-1) grid (5,9);
%	\draw[<->,thick] (-5,0) -- (5,0);
%	\draw[<->,thick] (0,-1) -- (0,9);
%	\draw[<->, ultra thick, blue, domain=-3:3] plot (\x,\x*\x);
%\end{tikzpicture}
%\end{minipage}
%
%Compare the rules: How are they the same? How are they different? Compare the graphs: How are they the same? How are they different?
%
%These are not the only three families of functions out there. Invent a rule or draw a graph which you believe \textit{does not fit} into any of these three families.
%\end{boxedexplore}
%
%The linear parent is $y=x$ and the quadratic parent is $y=x^2$. In this course, we will almost always use $y=2^x$ as the parent function for the exponential family, though other choices are possible.\footnote{A more natural choice for the exponential parent is the function $y=e^x$, where $e$ is Euler's number (yes, that's the same Euler as the function notation guy). Euler's number has many important connections to the family of exponential functions, but those will have to wait until later. By the way $e$, like $\pi$, is an irrational number: $e \approx 2.\, 71828\, 18284\, 59045\, 23536\, 02875\ldots$}
%
%\subsection{Fundamentals of Linear Functions}
%
%\subsubsection{Graph}
%
%The graph of a linear function is a straight, non-vertical line. (Can you explain why we have to exclude vertical lines from this family of functions?) We sometimes also exclude horizontal lines from this family: a function of the form $y = some~ number$ is a horizontal line, for example $y = 6$.
%
%\subsubsection{Equation}
%
%We will spend quite a bit of time studying linear functions in the coming chapters, and we will learn several different forms for the equation of a linear function. The feature they all share is that the highest power of the variable $x$ is 1. Most of the time, we don't write the exponent: $y = x$ is the same as $y = x^1$ (there's a phantom 1 up there in the exponent).
%
%In the extended exploration, we saw that graphing the equation $y=3x+2$ produced a graph that was steeper than $y=x$ and also shifted upwards from the origin. Generally, if take the equation $y=mx+b$ and replace $m$ and $b$ with numbers, the number $m$ controls the steepness of the line and the number $b$ controls the vertical shift.
%
%\subsubsection{Table}
%
%Recall that arithmetic sequences are in the family of linear functions. Arithmetic sequences have a rule that involves repeatedly adding on the constant difference at each step. So, this is the feature we see in the table of a linear function: when the $x$-values increase by a fixed amount, the $y$-values increase by a (possibly different) fixed amount.
%
%If we think of the rule $y=mx+b$ as the rule for an arithmetic sequence, we have the recursive rule ``start with $b$, add $m$ to the previous value''.
%
%In terms of a distance-time graph, this would be like moving by a fixed amount $m$ with each tick of the clock. In other words: moving at a constant speed. This gives us a good idea of why the graph is a straight line!
%
%%we wrote rules of the f: In the context of our work with sequences, $m$ was the common difference and $b$ was the $a_0$ term.
%%
%%ne of which is generally written $y = mx + b$. To create a specific linear function, we replace $m$ and $b$ with numbers.
%
%%Later, when we study linear functions we will refer to these as slopes and y-intercepts respectively.
%
%%You should have also noticed from the lesson on transformations in a plane that the ``b'' in y = mx + b caused the graph to translate up or down. Also, the ``m'' changed the steepness of the line and if it is negative, it changes the line from being increasing to decreasing (reflection through the x-axis).
%%y=x
%
%\subsection{Fundamentals of Exponential Functions}
%
%\subsubsection{Graph}
%The graph of an exponential function is a smooth, curving, J-like shape. It has a portion that becomes almost vertical and a portion that is almost horizontal.
%
%\subsubsection{Equation}
%
%
%
%Recall that geometric sequences are in the family of exponential functions. Geometric sequences have a rule that involves repeatedly multiplying by the constant ratio. This is the feature we look for in a table
%
%As we saw when studying Sierpi\'nski's carpet, multiplying causes values to make huge leaps in growth. After just eight iterations of the carpet procedure, the fractal was made up of over 16 million tiny squares! This gives us an idea of why the graph becomes so steep as $x$-values get larger.
%
%\subsubsection{Table}
%
%For negative $x$-values, we see the graph flatten out. Exponentials have the interesting property that while they get ``infinitely large'' (so to speak) on one side, they get ``infinitely small'' on the other side. In other words, we have tinier and tinier fractions that get closer and closed to zero\ldots\ but never actually disappear! Mathematically speaking, we say that the function has an \gls{asymptote} at the flat horizontal portion.
%
%MORE TO COME.
%
%%The general form is y = a*bx + c. For these equations, x is always in the exponent. Also, b can never be negative. That would cause the table values to alternate, giving a graph that does not follow the shape of the exponential family. To create a specific exponential function, replace a, b, and c with numbers. The recursive rule is always ``Start with a, multiply previous by b.''
%%
%%$y = 2^x$
%%Insert sample graph.
%%
%%The b was the common ratio. The Table will have a constant multiplier. Sometimes it will be hidden by a shift caused by addition or subtraction. If you actually calculated the Rate of change, you would notice that it is also exponential.
%%
%%For the transformations, c causes the graph to translate up or down. It also causes a shift in the asymptote. It is not the y-intercept. If you change b, or the common ratio, the graph can grow faster or slower or become decreasing instead of increasing (a reflection through the y-axis). If a is negative the graph will reflect through the x-axis.
%
%\subsection{Fundamentals of Quadratic Functions}
%
%\subsubsection{Graph}
%
%The graph of a quadratic function is a smooth U-shaped graph called a \gls{parabola}.
%
%\subsubsection{Equation}
%
%\subsubsection{Table}
%
%MORE TO COME.
%
%%Standard Form: y = ax2 + bx + c. For these equations, x is always raised to the second power. No higher exponents! A variable with a higher exponent takes priority in the equation and will dictate which family an equation belongs to. To create a specific quadratic function, replace a, b, and c with numbers. You don't have to have a b or c term, but you need an a, even if it is only 1. Tables will show a constant second difference. The recursive rule is always ``Start with A, add some arithmetic sequence to previous term.''
%
%%$y = x^2$
%%Insert sample graph.
%%
%%This standard form quadratic is different than the form used in the transformation investigation. The a in standard form changes the width of the parabola. If it is negative it makes the parabola have a maximum instead of a minimum. The c is not a move up or down. The c is the y-intercept. The ``bx'' term , if present, will cause a horizontal shift of the graph. The previous investigation we used (x-c)2 to perform a shift to the left or right. This is related to the ``bx'' term and you will learn more about how to do that horizontal shift when we study the quadratics in depth.
%
%
%% % % % % % % % % % % % % % % % % % % % % % % % % % % % % % % % % % % % % % % % 
%\section{Domain and Range}
%
%%\begin{boxedexplore}[Name of Extended Exploration]
%%\addtodoitem{Click here to visit the extended exploration: NAME}
%%\end{boxedexplore}
%
%Machines (meaning: real, mechanical devices) are physical things that have physical limitations. A blender, for instance, is a machine that has limited operating parameters. Anything we put into the blender comes out ``blended''. But, of course, there are some things we can't put in a blender: a piano, for example.\footnote{We'll pause here to note that ``blended'' is defined rather loosely: smoothies come out ``blended'', ice comes out ``crushed'', chickpeas come out ``pur\'eed'', magazines come out ``shredded''\ldots{} We consider all of these to be a kind of ``blended''. We'll also mention that there are things which one \textit{can} put into a blender, but \textit{shouldn't}.} When we think about number machines (meaning: functions) it often matters what kinds of input they can accept, and what kinds of output they can produce.
%
%\begin{boxedexplore}[Startup Exploration: The INs and OUTs of Functions]
%Consider the function machine $y = \abs{x}$. What sorts of values are allowed as input to this machine? What sorts of values will the machine produce as output?
%
%Consider the function machine $y = \frac{1}{x}$. What sorts of values are allowed as input to this machine? What sorts of values will the machine produce as output?
%\end{boxedexplore} %% End of startup exploration
%
%In the case of $y = \abs{x}$, we can put any number in this machine: whole numbers, fractions, positive numbers, negative numbers, it doesn't matter. But, only non-negative numbers every come out. So, we say that the domain of this function is ``all real numbers'' (any value is allowed as input), and that the range of this function is ``all non-negative real numbers''.
%
%\begin{boxeddef}[Domain]
%The set of all allowed input values to a function. This is the set of all allowed $x$-values, or all possible values of the independent variable.
%\end{boxeddef}
%
%\begin{boxeddef}[Range]
%The set of all output values from a function. This is the set of all $y$-values, or all possible values of the dependent variable.
%\end{boxeddef}
%
%%An important part of the study of functions is being able to determine the domain and range from the different representations.
%
%In the case of $y = \frac{1}{x}$, recall that division by zero is off-limits! So, this function can accept any nonzero number as input. We say that the range is ``all real numbers different from zero''. This function outputs only nonzero numbers, so it's range is ``all real numbers different from zero'' as well.
%
%
%\subsection{Domain and Range of the Three Families}
%
%It's super easy to identify the domains of the three main families of algebra 1. Linear, exponential, and quadratic functions all have ``all real numbers'' as their domain. We can always use any number as input.
%
%The range of linear functions is also pretty simple. It is all real numbers! As we saw as we transformed linear functions, the a (non-horizontal) line will eventually stretch up (or down) as high (or as low) as you want to go on the $y$-axis.
%
%The ranges for exponential and quadratic functions are a little more complicated. Perhaps the easiest way to determine the range is to look at the graph of the function. Notice, for example, that the graphs of $y=2^x$ and $y=x^2$ never dip below the $x$-axis. Neither one of these functions will ever give a negative number as output!
%
%\subsubsection{Range of an Exponential Function}
%
%...
%
%\subsubsection{Range of a Quadratic Function}
%
%A quadratic function will always be U-shaped, and always have either a highest or lowest point. This turning point is called the \textit{vertex} of the parabola, and it is the point that dictate their range of that particular function. 
%
%In the case of $y=x^2$, the vertex is at the origin (0,0) and it is the minimum value of the function. This means that the lowest possible $y$-value is 0. Therefore the range is $y \geq 0$.
%
%
%
%%, and if necessary, the table.
%%%
%%%Example 1
%%%State the Range of the given function: $y = 3^x$
%%%
%%%Solution:
%%%Exponential functions have asymptotes that will dictate their ranges. The asymptote is a line that the graph will get close to, but never really cross over.
%%%
%%%In this case the asymptote is the horizontal line y=0. This means that the y-values of this function will get really close to zero, but can't be zero.
%%%
%%%Therefore the range is $y > 0$.
%%%
%%%Example 2
%%%
%%%State the Range of the given function: $y = x^2$
%%%
%%%Solution:
%%%
%%%Graphs
%%%Sometimes you are given portions of graphs and asked to find domains and ranges. The key to doing this is correct interpretation of the end points. Open circle endpoints are exclusive and use < or >. Filled in circle endpoints are inclusive and use $\leq$ or $\geq$.
%%%
%%%Remember, the domain is the set all possible x-values and range are all possible y-values. Look at the axes when determining the domains and ranges. We will be writing the domains and ranges as combined inequalities.
%%%
%%%Example 3
%%%
%%%Problem: State the Domain and Range
%%%
%%%Solution:
%%%This graph is pretty straight forward because it is a line.
%%%
%%%Looking along the x-axis, you will notice that the domain spans -5 to 6. The endpoint at -5 is open and the one at 6 is.
%%%
%%%Looking along the y-axis, the range spans -3 to 4. -3 is closed and 4 is open.
%%%
%%%Domain: $-5 < x \leq 6$
%%%Range: $-3 \leq y < 4$
%%%
%%%Example 4
%%%
%%%Problem: State the Domain and Range
%%%
%%%Solution:
%%%This graph is has a curve that peaks above either end points. This will affect the range as the domain is identical to that of example 3.
%%%
%%%Looking along the y-axis, the range spans -3 to 7. -3 is closed and 7 is closed.
%%%
%%%Domain: $-6< x \leq 6$
%%%Range: $-3 \leq y \leq 7$
%%%
%%%Notice that I used the exact same end points for each of the graphs. That is to show you that it is not just the end points that determine the domain and range. You have to be aware of what happens between the end points, especially for the range.
%%%
%%%
%%%%\subsection{Sequences, Revisited}
%%%
%%%A sequence is a function whose domain is the set of natural numbers (the positive integers).
%%%
%%%Tables/Mappings/Sets
%%%
%%%If asked to find the domain or range from one of these representations, list the values in set notation.
%%%
%%%Example 5
%%%Problem: State the domain and range.
%%%
%%%x 	y
%%%5	14
%%%6	17
%%%7	20
%%%8	23
%%%
%%%Solution:
%%%Just list the x and y values in set notation.
%%%Domain = $\{5, 6, 7, 8\}$ Range = $\{14, 17, 20, 23\}$
%%%
%%%Limiting domain
%%%
%%%There are certain circumstances where you will want to limit your domain. For example, if you write an equation that shows distance as a function of time, you want to limit your domain to only include positive x-values.
%%%You can limit the function further by specifically defining your domain
%%%
%%%Example 6
%%%Given $y = 4x + 5$ and domain $x \in \{-1,0,1\}$ find the range.
%%%
%%%Solution:
%%%To find the range, substitute in the -1, 0, and then the 1 to find out what y is for each of those x values:
%%%
%%%f(-1) = 4(-1) + 5 = -4 + 5 = 1
%%%f(0) = 4(0) + 5 = 0 + 5= 5
%%%f(1) = 4(1) + 5 = 4 + 5 = 9 So the Range is {1, 5, 9}
%%%
%%%On graphing homework assignments when I ask you to use certain x-values, I'm limiting the domain.
%%%
%%%\subsection{Limited Domains}
%%%
%%%As mentioned above, three families of functions that we will study most closely in algebra 1 (the linear, exponential, and quadratic families) all have $\R$ as their domain.
%%%
%%%As we go along, however, we will encounter certain functions that have limited (or restricted) domains. Consider, for example, the rules
%%%\[y=\sqrt{x} \quad\text{and}\quad y = \frac{1}{x}\]
%%%There are certain values that cannot be allowed as input to these functions. In other words, we must exclude certain values from the domain because the rule will ``break'' if those values are used as input.
%%%
%%%For $y = \sqrt{x}$, we must enforce the rule that $x$ be greater than or equal to zero, because we can't take the square root of a negative number (yet). For $y \frac{1}{x}$, we know that $x$ cannot be equal to zero, because (as we saw in SOME SECTION) division by zero is an undefined operation.
%%%
%%%More on these types of functions later!
%%
%%
%%
%%
%
%
%%
%%\subsection{Transformations in the plane}
%%
%%As we saw in \cref{sec:mathrelationships}, the graph of a function is a particularly helpful for understanding the relationship described by the function. One of the skills we'll develop as we go forward in this course is the ability to connect features of the equation of a certain function to features of its graph.
%%
%%Over time and with practice, we'll get better at picturing the graph of a function in our mind's eye without having to draw the graph on paper. Drawing graphs will still be informative, of course, but some features of the graph will start to ``jump out at us'' simply from looking at the rule for the function.
%%
%%\begin{boxedexplore}[Transforming Graphs]
%%\addtodoitem{Click here to visit the extended exploration: Transforming Graphs}
%%\end{boxedexplore}
%%
%%\begin{boxedexplore}[Name of Startup Exploration]
%%Description of startup exploration.
%%\end{boxedexplore} %% End of startup exploration
%%
%%\subsection{Transformations of Functions}
%%
%%Geometrically speaking, a transformation is any ...
%%
%%The key transformations are:
%%\begin{itemize}
%%\item \textit{Translation}: sliding a figure up, down, left, or right
%%\item \textit{Reflection}: like in a mirror
%%\item \textit{Dilation}: stretching or shrinking a figure
%%\item \textit{Rotation}: Turning or twisting a figure
%%\end{itemize}
%%
%%Our focus in algebra 1 is on the first three types of transformations. We won't get into rotations of functions now. They will return later in your mathematical career!
%%
%%\subsubsection*{Translation Up or Down}
%%
%%The transformation $f(x) + b$ will move a graph up or down. If $b$ is positive, then the graph will move up. If $b$ is negative, then the graph will move down.
%%
%%\subsubsection*{Translation Right or Left}
%%
%%The transformation $f(x -c)$ will shift a graph left or right. If $c$ is positive, the graph will be shifted to the right, if $c$ is negative, the graph will be shifted to the left.
%%
%%\addtodoitem{Functions: could we use f(x+c) in the exploration, and have students notice the sign thing x+c versus x-c?}
%%
%%\subsubsection{Reflection Through the x-axis}
%%
%%The transformation $-f(x)$ is a reflection through the $x$-axis.
%%
%%\subsubsection{Dilations}
%%
%%The transformation $a*f(x)$ is a dilation. If $\abs{a}$ is greater than 1, the dilation is a contraction (lines become steeper, parabolas become thinner). If $\abs{a}$ is less than 1, the dilation is an expansion (lines become less steep, parabolas become wider).
%%
%%\addtodoitem{Functions: Transformation section needs some graphs and examples.}
