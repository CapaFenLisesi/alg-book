%%% Glossary definitions

\newglossaryentry{abscissa} {
	name=abscissa,
	description={The x-coordinate of a point in the coordinate plane.}
}

\newglossaryentry{absolute value} {
	name=absolute value,
	description={For real numbers, it is the distance a number is away from zero on a number line. It is a scalar quantity, meaning it just has a magnitude and no direction (sign). The absolute value of a number is always non-negative. In the order of 
operations, it works like a grouping symbol. The ``absolute value of x'' is denoted $\abs{x}$.}
}

\newglossaryentry{addition property of equality} {
	name=addition property of equality,
	description={For all real numbers $a$, $b$, and $c$: If $a = b$, then $a + c = b + c$. This axiom is used when solving equations. }
}

\newglossaryentry{addition property of order} {
	name=addition property of order,
	description={For all real numbers $a$, $b$, and $c$: If $a > b$, then $a + c > b + c$. This axiom is used when solving inequalities and also applies to inclusive 
symbols of order. }
}

\newglossaryentry{additive identity} {
	name=additive identity,
	description={The number which, when added to a given number $x$, leaves $x$ unchanged. In the real number system, 0 is the additive identity. The existence of the additive identity is a \gls{field axiom}.}
}

\newglossaryentry{additive inverse} {
	name=additive inverse,
	description={The number which, when added to a given number $x$, gives a sum of 0, the additive identity. The opposite of a number is its additive inverse. The existence of the additive inverse is a \gls{field axiom}.}
}

\newglossaryentry{algebraic expression} {
	name=algebraic expression,
	description={A symbolic representation of mathematical operations that can involve both numbers and variables. There is no equal sign in an expression.}
}

\newglossaryentry{algebraic number} {
	name=algebraic number,
	description={A number that is the root on a nonzero polynomial equation in one variable with rational coefficients. The set of algebraic numbers is a subset of the real numbers.}
}

\newglossaryentry{arithmetic sequence} {
	name=arithmetic sequence,
	description={A sequence where the difference between each pair of successive terms is constant. The constant difference is called the ``common difference'', usually denoted $d$.}
}

\newglossaryentry{associative property of addition} {
	name=associative property of addition,
	description={For all real numbers $a$, $b$, and $c$: $a + (b + c) = (a + b) + c$. This field axiom allows for the regrouping of longer strings of addition.}
}

\newglossaryentry{associative property of multiplication} {
	name=associative property of multiplication,
	description={For all real numbers $a$, $b$, and $c$: $a (b c) = (a b) c$. This field axiom allows for the regrouping of longer strings of multiplication.}
}

\newglossaryentry{asymptote} {
	name=asymptote,
	description={A line that a curve approaches as they both tend towards infinity. There are three types of asymptotes: vertical, horizontal, and oblique (slant). Exponential functions have a horizontal asymptote.}
}

\newglossaryentry{axiom} {
	name=axiom,
	description={A property or statement that is accepted without proof.}
}

\newglossaryentry{axis} {
	name=axis,
	plural=axes,
	description={One of two perpendicular number lines used to locate points in the coordinate plane. The plural form is ``axes''.}
}

\newglossaryentry{axis of symmetry} {
	name=axis of symmetry,
	plural=axes of symmetry,
	description={The line about which one can reflect an image onto itself. For example, a parabola has an axis of symmetry. Given the graph of a quadratic function, the axis of symmetry is a vertical line through the vertex. When written in standard form, the equation for the line of symmetry is given by $x = -\frac{b}{2a}$.}
}

\newglossaryentry{base} {
	name=base,
	description={(1) For triangles: A side of the triangle. (2) For expressions: A term or expression that is raised to a power.}
}

\newglossaryentry{binomial} {
	name=binomial,
	description={A polynomial with exactly 2 terms.}
}

\newglossaryentry{boundary} {
	name=boundary,
	description={For a one-variable \gls{inequality}, the boundary is a point on the number line. Inclusive boundaries are drawn as closed or filled in points, and exclusive boundaries are draw as open circles. For a two-variable inequality, the boundary is a line or curve. Inclusive boundaries are drawn as solid 
lines/curves, and exclusive boundaries are lines/curves drawn with a dashed or dotted line. For a linear inequalities, the boundary line separates the plane into two \glspl{half-plane}, one of which will contain the solutions to the inequality.}
}

\newglossaryentry{Cartesian plane} {
	name=Cartesian plane,
	description={See \gls{coordinate plane}.}
}

\newglossaryentry{closure} {
	name=closure,
	description={A set is said to be to ``have closure'' (or to ``be closed'') under an operation performing the operation on members from the set always yields a result that is also a member of the set. The \glspl{natural number}, for example, are closed under the operation of addition, since the sum of any two natural numbers is itself a natural number. The natural numbers are not closed under the operation of subtraction.}
}

\newglossaryentry{coefficient} {
	name=coefficient,
	description={The numerical factor in a term with a variable. If the number is not explicitly written, the coefficient is understood to be 1.}
}

\newglossaryentry{colinear} {
	name=colinear,
	description={To be on the same line.}
}

\newglossaryentry{combining like terms} {
	name=combining like terms,
	description={A short-cut used to add terms that have exactly the same variables raised to the same exponents.}
}

\newglossaryentry{common monomial factor} {
	name=common monomial factor,
	description={A monomial that is a factor of every term in a polynomial expression.}
}

\newglossaryentry{common difference} {
	name=common difference,
	description={In an arithmetic sequence, it is the constant difference between successive terms.}
}

\newglossaryentry{common ratio} {
	name=common ratio,
	description={In an geometric sequence, it is the constant ratio between successive terms.}
}

\newglossaryentry{commutative property of addition} {
	name=commutative property of addition,
	description={For all real numbers $a$ and $b$: $a + b = b + a$. This field axiom allows for the reordering longer strings of addition. }
}

\newglossaryentry{commutative property of multiplication} {
	name=commutative property of multiplication,
	description={For all real numbers $a$ and $b$: $a b = b a$. This field axiom allows for the reordering longer strings of multiplication. }
}

\newglossaryentry{completing the square} {
	name=completing the square,
	description={Using the properties of equality on a quadratic equation to convert one side into a perfect square trinomial. Completing the square can be used as a 
technique to solve quadratic equations.}
}

\newglossaryentry{complex number} {
	name=complex number,
	description={A member of the set of numbers that consists of real and imaginary numbers. The set is denoted $\C$.}
}

\newglossaryentry{compound interest} {
	name=compound interest,
	description={A way to calculate interest based on both the principal amount and any interest already accrued. This type of interest is an exponential relationship. The formula is $A = P \left(1 + \frac{r}{n} \right)^{nt}$, where $P$ is the principal amount, $r$ is the rate of interest, $t$ is the amount of time over which interest is to be computed, and $n$ is the number of compounding periods per unit of time.}
}

\newglossaryentry{constant} {
	name=constant,
	description={A value that does not change.}
}

\newglossaryentry{constant function} {
	name=constant function,
	description={A function whose graph is a horizontal line. It is of the form $f(x) = c$, where $c$ is a constant. Constant functions are polynomial functions of degree zero.}
}

\newglossaryentry{constant multiplier} {
	name=constant multiplier,
	description={In a sequence that grows or decays exponentially, the number each term is multiplied by to get the next term. Also known as the ``common multiplier'', or ``common ratio''.}
}

\newglossaryentry{constant of variation} {
	name=constant of variation,
	description={The constant ratio in a direct variation or the constant product in an inverse variation. It is designated with the variable $k$.}
}

\newglossaryentry{constant term} {
	name=constant term,
	description={A term that includes no variable.}
}

\newglossaryentry{constraint} {
	name=constraint,
	description={The limitations on the values of the variables in a problem. Equations, inequalities and systems are used model the constraints in real-world situations.}
}

\newglossaryentry{continuous data} {
	name=continuous data,
	description={Data that has no breaks and has measurements that can change between data points. Graphically, the measured data points are connected with lines or curves.}
}

\newglossaryentry{continuous function} {
	name=continuous function,
	description={A function that has no breaks in the domain or range. The graph of a continuous function is a line or curve with no holes, gaps, or vertical asymptotes.}
}

\newglossaryentry{converse of the Pythagorean theorem} {
	name=converse of the Pythagorean theorem,
	description={If a triangle has sides $a$, $b$, and $c$, such that $a^2+b^2=c^2$, then the triangle is a right triangle with a hypotenuse of length $c$.}
}

\newglossaryentry{conversion factor} {
	name=conversion factor,
	description={A ratio used to convert measurement from one unit to another.}
}

\newglossaryentry{coordinate plane} {
	name=coordinate plane,
	description={A plane with a pair of scaled, perpendicular axes allowing one to locate points with ordered pairs and to represent lines and curves by equations. Also known as the Cartesian plane, named for its creator, French philosopher Ren\'{e} Descartes.}
}

\newglossaryentry{coprime} {
	name=coprime,
	description={See \gls{relatively prime}.}
}

\newglossaryentry{correlation} {
	name=correlation,
	description={Used in describing data graphed in a scatter plot. It is a trend between two variables. A trend can show positive, negative, or no correlation. Positive correlation shows an \gls{increasing} trend in data. Negative correlation shows a \gls{decreasing} trend in data.}
}

\newglossaryentry{cubic} {
	name=cubic,
	description={A function, number, or expression raised to the third power. Called cubic as it relates to the volume of a cube.}
}

\newglossaryentry{decay factor} {
	name=decay factor,
	description={In exponential decay, the constant multiplier used to calculate the amount of decay after each unit of time. In the formula $y = (1-r)^x$, it is the quantity $(1-r)$. It represents the quantity remaining and is the common multiplier in the exponential relationship.}
}

\newglossaryentry{decay rate} {
	name=decay rate,
	description={In exponential decay, the fraction or percentage by which a population decreases for each unit of time. In the formula $y = (1-r)^x$, it is the quantity $r$.}
}

\newglossaryentry{decreasing} {
	name=decreasing,
	description={A function is said to be decreasing if as $x$ increases, $y$ decreases. Lines with negative slopes are decreasing.}
}

\newglossaryentry{degree of a polynomial} {
	name=degree of a polynomial,
	description={The degree of the term in a polynomial with the highest degree.}
}

\newglossaryentry{degree of a term} {
	name=degree of a term,
	description={The power (exponent) to which the variable is raised in a variable term. If there is no exponent explicitly written on a variable in a term, the term is understood to be of degree 1. The degree of a constant term is zero.}
}

\newglossaryentry{delta}
{
	name={$\Delta$},
	description={Delta, the fourth letter of the Greek alphabet. Used to represent change. They symbol $\Delta x$ is read ``delta $x$'' or ``the change in $x$''.}
}

\newglossaryentry{denominator} {
	name=denominator,
	description={The number or expression below the \gls{vinculum} in a \gls{rational number} or \gls{rational expression}. For example, in the number $\frac{5}{2}$, the denominator is 2.}
}

\newglossaryentry{dependent variable} {
	name=dependent variable,
	description={A variable whose values depend on the values of another variable. In a graph of the relationship between the two variables, the values on the vertical axis represent the values of the dependent variable. The generic variable used is $y$.}
}

\newglossaryentry{difference of squares} {
	name=difference of squares,
	description={A binomial of the form $a^2-b^2$.}
}

\newglossaryentry{dimensional analysis} {
	name=dimensional analysis,
	description={A strategy for converting a measurement from one unit to another using multiplication by a string of conversion factors. The key is to include the units with the numbers. It is used often in science.}
}

\newglossaryentry{direct variation} {
	name=direct variation,
	description={In Algebra 1, a relationship in which the ratio of two variables is constant. A direct variation has an equation of the form $y = kx$. The quantities represented by $x$ and $y$ are said to be \gls{directly proportional}. The value $k$ is called the \gls{constant of variation}.}
}

\newglossaryentry{directly proportional} {
	name=directly proportional,
	description={Used to describe two variables whose values have a constant ratio.}
}

\newglossaryentry{discrete data} {
	name=discrete data,
	description={Data that can only take on certain values. Discrete data usually involves a count of items.}
}

\newglossaryentry{discrete function} {
	name=discrete function,
	description={A function whose domain and range have breaks or are made up of distinct values rather than intervals of real numbers. The graph of a discrete function will have breaks or will be made up of distinct points.}
}

\newglossaryentry{discriminant} {
	name=discriminant,
	description={The expression under the square root in the quadratic formula, used to determine the number and nature of the roots of a quadratic. If a quadratic equation is written in standard form, then the discriminant is $b^2 - 4ac$. If the value of the discriminant is greater than 0, there are two real solutions to the quadratic equation. If it is equal to zero, there is one real solution. If it is less than zero, there are no real solutions to the quadratic equation.}
}

\newglossaryentry{distance formula} {
	name=distance formula,
	description={A formula based on the Pythagorean theorem that uses the coordinates of two points to calculate the distance between the two points. The formula for the distance $d$ between any two points $(x_1, y_1)$ and $(x_2, y_2)$ is $d = \sqrt{ (x_2 - x_1)^2 + (y_2 - y_1)^2 }$.}
}

\newglossaryentry{distributive property} {
	name=distributive property,
	description={For all real numbers $a$, $b$, and $c$: $a (b + c) = ab + ac$. This field axiom allows one to simplify an expression without having the evaluate the sum inside the grouping symbol first.}
}

\newglossaryentry{division property of equality} {
	name=division property of equality,
	description={For all real numbers $a$, $b$, and $c$, where $c \neq 0$: if $a = b$, then $\frac{a}{c} = \frac{b}{c}$. This property is a version of the multiplication property of equality. It is used when solving equations.}
}

\newglossaryentry{division property of order} {
	name=division property of order,
	description={For all real numbers $a$, $b$, and $c$ where $c>0$: if $a < b$, then $\frac{a}{c} < \frac{b}{c}$. If, on the other hand, $c<0$, then $a < b$ implies $\frac{a}{c} > \frac{b}{c}$. This property is a version of the multiplication property of order. It is used when solving inequalities.}
}

\newglossaryentry{domain} {
	name=domain,
	description={The set of all input values of a function, or the $x$-values. In a problem context it is represented by the independent variable.}
}

\newglossaryentry{domain restriction} {
	name=domain restriction,
	description={Values that cannot be used in the domain of a function. Radical and rational functions have domain restrictions.}
}

\newglossaryentry{doubling time} {
	name=doubling time,
	description={In exponential growth, the amount of time it takes for a population, or amount, to double in size. It is constant for an exponential relationship.}
}

\newglossaryentry{elimination method} {
	name=elimination method,
	description={A method for solving a system of equations that involves adding or subtracting multiples of the equations in order to eliminate a variable. It is based on Gaussian Elimination, a method to solve systems of equations that have been converted into matrices.}
}

\newglossaryentry{equation} {
	name=equation,
	description={A statement that says the value of one expression is the same as the value of another expression.}
}

\newglossaryentry{equivalent equations} {
	name=equivalent equations,
	description={Equations that have the same solution set.}
}

\newglossaryentry{equivalent inequalities} {
	name=equivalent inequalities,
	description={Inequalities that have the same solution set.}
}

\newglossaryentry{evaluate} {
	name=evaluate,
	description={To find the value of an expression. If an expression contains variables, values must be substituted for the variable before the expression can be evaluated.}
}

\newglossaryentry{exclusive boundary} {
	name=exclusive boundary,
	description={See \gls{boundary}.}
}

\newglossaryentry{exclusive inequality} {
	name=exclusive inequality,
	description={See \gls{inequality}.}
}

\newglossaryentry{exponent} {
	name=exponent,
	description={A number or variable written as a small superscript to a number or a variable, called the \gls{base}, that indicates how many times the base is being used as a factor.}
}

\newglossaryentry{exponential decay} {
	name=exponential decay,
	description={A decreasing pattern in which amounts decrease by a constant percent.}
}

\newglossaryentry{exponential equation} {
	name=exponential equation,
	description={An equation in which a variable appears in the exponent.}
}

\newglossaryentry{exponential form} {
	name=exponential form,
	description={The form of an expression in which repeated multiplication is written using exponents.}
}

\newglossaryentry{exponential function} {
	name=exponential function,
	description={A function that repeatedly multiplies an initial amount by the same positive number. They can all be modeled using $y = ab^x$ where $a$ is the initial amount and $b$ is the constant multiplier.}
}

\newglossaryentry{exponential growth} {
	name=exponential growth,
	description={An increasing pattern in which amounts increase by a constant percent.}
}

\newglossaryentry{extraneous solution} {
	name=extraneous solution,
	description={An apparent solution of an equation that does not satisfy the original equation. They occur when the transformation of an equation changes the solution set of the original equation, for example squaring both sides of an equation or multiplying by a quantity that can be zero.}
}

\newglossaryentry{factor} {
	name=factor,
	description={One of the numbers, variables, or expressions multiplied to obtain a product.}
}

\newglossaryentry{factored form} {
	name=factored form,
	description={The form of an expression when it is written as the product of factors. The factors can be numbers, variables, or expressions. Factored form is not simplified.}
}

\newglossaryentry{factoring} {
	name=factoring,
	description={The process of rewriting an expression as a product of factors.}
}

\newglossaryentry{family of functions} {
	name=family of functions,
	plural=families of functions,
	description={Similar functions that are all transformations of the same parent function.}
}

\newglossaryentry{field axiom} {
	name=field axiom,
	description={One of a set of axioms including closure, identity, inverse, associative, commutative, and distributive properties. Along with a few definitions and properties of equality, they create the foundation upon which algebra is built.}
}

\newacronym{FOIL}{FOIL}{A mnemonic for remembering the procedure to multiply two binomials. F stands for multiplying the first term in each binomial. O stands for multiplying the outer terms of the binomials. I stands for 
multiplying the inner terms of the binomials. L stands for multiplying that last 
term in each binomial.}

\newglossaryentry{fractal} {
	name=fractal,
	description={A geometric figure that has undergone infinite applications of a recursive procedure and which exhibits the property of self-similarity. }
}

\newglossaryentry{function} {
	name=function,
	description={A relation in which there is exactly one output value for each input value. The graph of a function must pass the vertical line test.}
}

\newglossaryentry{function notation} {
	name=function notation,
	description={A notation in which a function is named by a letter and the input is shown in parenthesis after the function name, generically, $f(x)$, read ``$f$ of $x$''. The variables used may be changed to better represent quantities in a problem, for example $d(t)$ may represent distance $d$ as a function of time $t$. When graphing in the \gls{coordinate plane}, $f(x)$ is another way to write $y$. When $x$ is replaced by a number, it indicates that one should evaluate the function at that value. The notation was first used by Swiss mathematician Leonhard Euler.}
}

\newglossaryentry{function rule} {
	name=function rule,
	description={An expression that represents the relationship between the variables of a function.}
}

\newacronym{GCF}{GCF}{Greatest Common Factor}

\newglossaryentry{geometric sequence} {
	name=geometric sequence,
	description={A sequence where the ratio between each pair of successive terms is constant. The constant ratio is called the ``common ratio'', usually denoted $r$. Geometric sequences are exponential.}
}

\newglossaryentry{growth factor} {
	name=growth factor,
	description={In exponential growth, the constant multiplier used to calculate the amount of growth after each unit of time. In the formula $y = (1+r)^x$, it is the quantity $(1+r)$. It is the common multiplier in the exponential relationship.}
}

\newglossaryentry{growth rate} {
	name=growth rate,
	description={In exponential growth, the fraction or percentage by which a population increases for each unit of time. In the formula $y = (1+r)^x$, it is the quantity $r$.}
}

\newglossaryentry{half-life} {
	name=half-life,
	description={The time needed for an amount of a substance to exponentially decay to half the original amount. Half-life is constant for an exponential relationship.}
}

\newglossaryentry{half-plane} {
	name=half-plane,
	description={The set of points on a plane that fall on one side of a boundary line. Part of the solution of a linear inequality in two variables is a half-plane.}
}

\newglossaryentry{hypotenuse} {
	name=hypotenuse,
	description={The side of a right triangle opposite the right angle. It is the longest side of the triangle.}
}

\newglossaryentry{identity} {
	name=identity,
	description={When solving equations with variables on both sides, identities occur when the equation is true for every value of the variable. The solution set $S$ is written as $S=\R$.}
}

\newglossaryentry{identity property of addition} {
	name=identity property of addition,
	description={The sum of any number and 0 is that number. For every real number $a$, $a + 0 = a$ and $0 + a = a$. The existence of the \gls{additive identity} is a \gls{field axiom}.}
}

\newglossaryentry{identity property of multiplication} {
	name=identity property of multiplication,
	description={The product of any number and 1 is that number. For every real number $a$, $a \cdot 1 = a$ and $1 \cdot a = a$. The existence of the multiplicative identity is a \gls{field axiom}.}
}

\newglossaryentry{imaginary number} {
	name=imaginary number,
	description={A member of the set of numbers that is created by taking the square root of a negative number. In the set of imaginary numbers, the square root of -1 is represented by the letter $i$. The set of imaginary numbers is a subset of the complex number system. The sets of real and imaginary numbers are disjoint, meaning they have no common members.}
}

\newglossaryentry{implied operation} {
	name=implied operation,
	description={An operation that is not explicitly written. For example, in $3 (x + 4)$ the multiplication between 3 and $(x + 4)$ is an implied operation, since no multiplication symbol is explicitly written in between.}
}

\newglossaryentry{improper fraction} {
	name=improper fraction,
	description={A fraction whose \gls{numerator} is greater than its \gls{denominator}. For example, $\frac{5}{2}$ is an improper fraction. A fraction that is not an improper fraction is called a \gls{proper fraction}. See also \gls{mixed number}.}
}

\newglossaryentry{increasing} {
	name=increasing,
	description={A function is said to be decreasing if as $x$ increases, $y$ increases. Lines with positive slopes are decreasing.}
}

\newglossaryentry{independent variable} {
	name=independent variable,
	description={A variable whose values affect the values of another variable. In a graph of the relationship between the two variables, the values on the horizontal axis represent the values of the dependent variable. The generic variable used is $x$.}
}

\newglossaryentry{inequality} {
	name=inequality,
	description={A statement that one quantity is less than or greater than another. An inequality may exclusive or inclusive. The exclusive inequalities are $<$ and $>$, read ``less than'' and ``greater than''. The inclusive inequalities are $\leq$ and $\geq$, read ``less than or equal to'' and ``greater than or equal to''.}
}

\newglossaryentry{initial value} {
	name=initial value,
	description={The starting value of a sequence or exponential function.}
}

\newglossaryentry{integer} {
	name=integer,
	description={A member of the set of natural numbers, their opposites, and zero. The set is denoted $\Z$, and we may write $\Z = \{0, \pm1, \pm2, \pm3, \dotsc \}$. The integers are a subset of the rational numbers.}
}

\newglossaryentry{inclusive boundary} {
	name=inclusive boundary,
	description={See \gls{boundary}.}
}

\newglossaryentry{inclusive inequality} {
	name=inclusive inequality,
	description={See \gls{inequality}.}
}

\newglossaryentry{intercept} {
	name=intercept,
	description={The point which a graph intersects one of the axes.}
}

\newglossaryentry{interest} {
	name=interest,
	description={A percentage of the balance added to an account at regular time intervals.}
}

\newglossaryentry{interest rate} {
	name=interest rate,
	description={The percentage used to calculate interest.}
}

\newglossaryentry{inverse property of addition} {
	name=inverse property of addition,
	description={For any real number $a$, there exists a real number $\umin a$ such that $a + \umin a = 0$. The number $\umin a$ is called the \gls{additive inverse} of $a$. Very often we will call it the \gls{opposite} of $a$.}
}

\newglossaryentry{inverse property of multiplication} {
	name=inverse property of multiplication,
	description={For any nonzero real number $a$, there exists a real number $\frac{1}{a}$ such that $a \cdot \frac{1}{a} = 1$. The number $\frac{1}{a}$ is called the \gls{multiplicative inverse} of $a$. Very often we will call it the \gls{reciprocal} of $a$.}
}


\newglossaryentry{inverse variation} {
	name=inverse variation,
	description={In Algebra 1, a relationship in which the product of two variables is constant. An inverse variation has an equation in the form $xy = k$, or $y = \frac{k}{x}$. The quantities represented by $x$ and $y$ are said to be \gls{inversely proportional}. The value $k$ is called the \gls{constant of variation}.}
}

\newglossaryentry{inversely proportional} {
	name=inversely proportional,
	description={Used to describe two variables whose values have a constant product.}
}

\newglossaryentry{irrational number} {
	name=irrational number,
	description={A number that cannot be expressed as the ratio of two integers. In decimal form, an irrational number has an infinite number of digits and does not repeat. The set of irrational numbers consist of algebraic and transcendental numbers. The set of irrational numbers is a subset of the real numbers.}
}

\newglossaryentry{irreversible operation} {
	name=irreversible operation,
	description={An operation performed when solving an equation that changes the solution set of the equation. Multiplying or dividing both sides of an equation by an expression that might equal zero are considered irreversible operations.}
}

\newglossaryentry{leg} {
	name=leg,
	description={One of the perpendicular sides of a right triangle.}
}

\newglossaryentry{like terms} {
	name=like terms,
	description={Terms with exactly the same variable factors in a variable expression. The variables and the powers to which the variables are raised must be identical for the terms to be considered like terms.}
}

\newglossaryentry{limited domain} {
	name=limited domain,
	description={The restricted domain of a function. Domains are usually limited in real world contexts. For example, we rarely allow negative values for a variable that represents ``time''. For this reason it is often referred to as a reasonable domain.}
}

\newglossaryentry{line of best fit} {
	name=line of best fit,
	description={A line used to model a set of data. A line of best fit shows general direction of the data. When hand-drawn, one should have about the same number of data points above and below the line. When using the linear regression tool on the 
calculator, the correlation coefficient will show how well the line fits the data.}
}

\newglossaryentry{line of symmetry} {
	name=line of symmetry,
	description={See \gls{axis of symmetry}.}
}

\newglossaryentry{linear} {
	name=linear,
	description={In the shape of a line or represented by a line. In mathematics, a linear equation or expression has variables raised only to the power of 1.}
}

\newglossaryentry{linear function} {
	name=linear function,
	description={A function characterized by a constant rate of change. The graph of a linear function is a non-vertical line. It is a polynomial of degree one.}
}

\newglossaryentry{linear inequality} {
	name=linear inequality,
	description={An inequality of two variables whose boundary is formed by a linear function. It describes a region of the coordinate plane that consists of a boundary line and a half-plane.}
}

\newglossaryentry{linear programming} {
	name=linear programming,
	description={A method to optimize a quantity that uses an objective function to represent the quantity and a system of linear inequalities to represent the constraints on the variables involved. The system of inequalities are graphed to represent a set of feasible solutions and the vertices of the region will describe the optimal amount of the quantity.}
}

\newglossaryentry{linear relationship} {
	name=linear relationship,
	description={A relationship that can be represented by a linear function. A linear relationship is characterized by a constant rate of change.}
}

\newglossaryentry{linear term} {
	name=linear term,
	description={A term of degree 1.}
}

\newglossaryentry{lowest terms} {
	name=lowest terms,
	description={The form of a fraction in which the numerator and denominator are \gls{relatively prime}. A fraction in lowest terms is also called a reduced fraction.}
}

\newglossaryentry{mapping diagram} {
	name=mapping diagram,
	description={A diagram used to determine if a relation is a function. The values of the domain and range are written in circles. Arrows are drawn from the elements of the domain to the corresponding elements of the range. It is a visual that shows 
how the members of the domain map to the members of the range.}
}

\newglossaryentry{mathematical equivalence} {
	name=mathematical equivalence,
	description={The idea that numbers, expressions, equations, functions, or other mathematical objects can be algebraically manipulated, using specific rules, such that their representations and appearance are changed while other fundamental properties remain unchanged.}
}

\newglossaryentry{mathematical modeling} {
	name=mathematical modeling,
	description={Translating a real-world scenario with a given set of constraints into an abstract representation that can be manipulated and studied mathematically. For example, creating a set of variables and equations to solve a \gls{linear programming} problem.}
}

\newglossaryentry{maximum} {
	name=maximum,
	description={The greatest value. In a quadratic function, the vertex will be a maximum if the coefficient of the quadratic term is negative.}
}

\newglossaryentry{midpoint} {
	name=midpoint,
	description={The point on a line segment halfway between the endpoints. The coordinates of the midpoint are found by averaging the abscissas and ordinates of the endpoints.}
}

\newglossaryentry{midpoint formula} {
	name=midpoint formula,
	description={The formula that can be used to compute the midpoint of a line segment. Given a line segment with endpoints $(x_1, y_1)$ and $(x_2, y_2)$, the midpoint of the segment has coordinates $\left( \frac{x_1+x_2}{2}, \frac{y_1+y_2}{2} \right)$.}
}

\newglossaryentry{minimum} {
	name=minimum,
	description={The smallest value. In a quadratic function, the vertex will be a minimum if the coefficient of the quadratic term is positive.}
}

\newglossaryentry{mixed number} {
	name=mixed number,
	description={The sum of a nonzero \gls{integer} and a \gls{proper fraction}. For example $2\frac{3}{5}$ is a mixed number. See also \gls{improper fraction}.}
}

\newglossaryentry{monomial} {
	name=monomial,
	description={A polynomial with only one term.}
}

\newglossaryentry{multiplication property of equality} {
	name=multiplication property of equality,
	description={For all real numbers $a$, $b$, and $c$: if $a = b$ then $ac = bc$. This property is used to solve equations.}
}

\newglossaryentry{multiplication property of order} {
	name=multiplication property of order,
	description={For all real numbers $a$, $b$, and $c$ and $c > 0$: if $a < b$ then $ac < bc$. If, on the other hand, $c < 0$, then $a < b$ implies $ac > bc$. This property is used to solve equations.}
}

\newglossaryentry{multiplicative identity} {
	name=multiplicative identity,
	description={The number which, when multiplied by a given number $x$, leaves $x$ unchanged. In the real number system, 1 is the multiplicative identity. The existence of the multiplicative identity is a \gls{field axiom}.}
}

\newglossaryentry{multiplicative inverse} {
	name=multiplicative inverse,
	description={The number which, when multiplied by a given nonzero number $x$, gives a product of 1, the multiplicative identity. The reciprocal of a number is its multiplicative inverse. The existence of the multiplicative inverse is a \gls{field axiom}.}
}

\newglossaryentry{natural number} {
	name=natural number,
	description={A member of the set $\{1, 2, 3, 4, \dotsc\}$, denoted $\N$. Also called the counting numbers. The number 0 is sometimes included as a natural number.}
}

\newglossaryentry{negative correlation} {
	name=negative correlation,
	description={See \gls{correlation}.}
}

\newglossaryentry{null set} {
	name=null set,
	description={A set that contains no elements. Also called the empty set. Used to show that there is no solution to an equation. Denoted $\emptyset$ or $\{ \}$.}
}

\newglossaryentry{numerator} {
	name=numerator,
	description={The number or expression above the \gls{vinculum} in a \gls{rational number} or \gls{rational expression}. For example, in the number $\frac{5}{2}$, the numerator is 5.}
}

\newglossaryentry{numeric expression} {
	name=numeric expression,
	description={An expression containing only numbers and mathematical operations.}
}

\newglossaryentry{obelus} {
	name=obelus,
	symbol={$\div$},
	description={The division symbol $\div$.}
}

\newglossaryentry{one-variable data} {
	name=one-variable data,
	description={Data that measures only one trait or quantity. A one-variable data set consists of single values (as opposed to ordered pairs) and is graphed on a number line. Compare with: \gls{two-variable data}.}
}

\newglossaryentry{opposite} {
	name=opposite,
	description={See \gls{additive inverse}.}
}

\newglossaryentry{optimization} {
	name=optimization,
	description={To maximize or minimize a quantity given constraints. For example a company will want to optimize (maximize) their profits while faced with constraints such as the cost and availability of labor and materials.}
}

\newglossaryentry{order of magnitude} {
	name=order of magnitude,
	description={A way of expressing the size of an very large or very small number by giving the power of 10 associated with the number.}
}

\newglossaryentry{order of operations} {
	name=order of operations,
	description={The agreed-upon order in which operations are carried out when evaluating an expression.}
}

\newglossaryentry{ordered pair} {
	name=ordered pair,
	description={A pair of numbers named in an order that matters. The coordinates of a point are given as an ordered pair in which the first number is the $x$-coordinate (abscissa) and the second number is the $y$-coordinate (ordinate).}
}

\newglossaryentry{ordinate} {
	name=ordinate,
	description={The y-coordinate of a point in the coordinate plane.}
}
  
\newglossaryentry{origin} {
	name=origin,
	description={The point where the coordinate axes intersect. In a coordinate plane it has the coordinates $(0,0)$.}
}

\newglossaryentry{parabola} {
	name=parabola,
	description={The set of all points whose distance from a fixed point (called the focus) is equal to the distance from a fixed line (called the directrix). Also known as the smooth ``U'' shaped curve of a quadratic function.}
}

\newglossaryentry{parallel lines} {
	name=parallel lines,
	description={Lines in the same plane that never intersect. They are always the same distance apart in Euclidean geometry. The slopes of parallel lines are the same.}
}

\newglossaryentry{parent function} {
	name=parent function,
	description={The most basic form of a function. A parent function can be transformed to create a family of functions.}
}

\newglossaryentry{percent change} {
	name=percent change,
	description={The percent by which an amount differs from its original amount. It is calculated by taking the amount of the change and dividing it by the original amount.}
}

\newglossaryentry{perfect cube} {
	name=perfect cube,
	description={A number that is equal to the cube of an integer, or a polynomial that is equal to the cube of another polynomial.}
}

\newglossaryentry{perfect square} {
	name=perfect square,
	description={A number that is equal to the square of an integer, or a polynomial that is equal to the square of another polynomial.}
}

\newglossaryentry{perfect square trinomial} {
	name=perfect square trinomial,
	description={A trinomial generated by squaring a binomial. For example, squaring the binomial $(a+b)$ yields $(a+b)^2 = a^2 + 2ab + b^2$. Thus, $a^2 + 2ab + b^2$ is a perfect square trinomial.}
}

\newglossaryentry{period of compounding} {
	name=period of compounding,
	description={The number of times interest is calculated during a year for compound interest. It is represented by $n$ in the \gls{compound interest} formula.}
}

\newglossaryentry{perpendicular lines} {
	name=perpendicular lines,
	description={Lines that intersect at a right angle. The slopes of perpendicular lines are opposites and reciprocals. The slopes of perpendicular lines multiply to $-1$.}
}

\newglossaryentry{point-slope form} {
	name=point-slope form,
	description={The form of a linear equation that uses the slope and any point on the line. It is written either $y-y_1 = m(x-x_1)$ or $y=m(x-x_1)+y_1$, where $m$ is the slope of the line and $(x_1,y_1)$ is a point on the line. It can be derived from the slope formula and represents the transformation of the line $y = mx$ where a vertical shift of $y_1$ and a horizontal shift of $x_1$ has occurred.}
}

\newglossaryentry{polynomial} {
	name=polynomial,
	description={A sum of terms that have positive integer exponents. In Algebra 1, all polynomials are in one variable.}
}

\newglossaryentry{positive correlation} {
	name=positive correlation,
	description={See \gls{correlation}.}
}

\newglossaryentry{principal amount} {
	name=principal amount,
	description={The original amount invested in a situation that involves accumulating interest. It is represented by $P$ in the \gls{compound interest} and simple interest formulas.}
}

\newglossaryentry{principal square root} {
	name=principal square root,
	description={The positive square root of a number.}
}

\newglossaryentry{proper fraction} {
	name=proper fraction,
	description={A fraction whose \gls{numerator} is less than its \gls{denominator}. For example, the fraction $\frac{7}{9}$ is a proper fraction. A fraction that is not a proper fraction is called an \gls{improper fraction}.}
}

\newglossaryentry{proportion} {
	name=proportion,
	description={An equation stating that two ratios are equal.}
}

\newglossaryentry{power} {
	name=power,
	description={An expression of the form $a^n$ is called a power of $a$.}
}
 
\newglossaryentry{Pythagorean theorem} {
	name=Pythagorean theorem,
	description={A formula that expresses the relationship between the sides of a right triangle. It states that the sum of the squares of the legs of a right triangle is equal to the square of the \gls{hypotenuse}.}
}

\newglossaryentry{quadrant} {
	name=quadrant,
	description={One of the four regions that a coordinate plane is divided into by the two axes. The quadrants are numbered I, II, III, and IV, starting in the upper right and moving counterclockwise.}
}

\newglossaryentry{quadratic formula} {
	name=quadratic formula,
	description={The formula used to find the exact solution to any quadratic equation. Given that $ax^2+bx+c=0$, the formula states \[x = \frac{-b \pm \sqrt{b^2 - 4ac}}{2a}\] It is derived by completing the square on the standard form quadratic equation.}
}

\newglossaryentry{quadratic function} {
	name=quadratic function,
	description={A function with an equation of the form $y = ax^2 + bx + c$ where $a \neq 0$. The graph of a quadratic function is a \gls{parabola}.}
}

\newglossaryentry{quadratic term} {
	name=quadratic term,
	description={A term of degree 2.}
}

\newglossaryentry{radical} {
	name=radical,
	description={The root symbol $\sqrt{~}$, used to denote square roots, cube roots, and so on. The symbol $\sqrt[n]{x}$ is read ``nth root of x.'' If $n$ is not stated, as in $\sqrt{x}$, it is understood to be 2 and the radical indicates the square root.}
}

\newglossaryentry{radical expression} {
	name=radical expression,
	description={An expression containing a radical (square root, cube root, or any $n$th root).}
}

\newglossaryentry{radical function} {
	name=radical function,
	description={A function where the independent variable is under a radical (square root, cube root, or any $n$th root).}
}

\newglossaryentry{radioactive decay} {
	name=radioactive decay,
	description={The process by which an unstable element loses mass with a release of energy, transforming it into a different element or isotope.}
}

\newglossaryentry{range} {
	name=range,
	description={(1) In statistics it is the difference between the greatest value in a data set and the smallest value in a data set. (2) In the study of functions it is the set of all output values of a function. It is represented by the dependent variable.}
}

\newglossaryentry{rate} {
	name=rate,
	description={A \gls{ratio} that measures two quantities with different units.}
}

\newglossaryentry{rate of change} {
	name=rate of change,
	description={A measurement of how quickly one quantity changes relative to another quantity. Given values $(x_1,y_1)$ and $(x_2,y_2)$, the rate of change of $y$ with respect to $x$ is $\frac{\Delta y}{\Delta x} = \frac{y_2-y_1}{x_2-x_1}$. The patterns of the rate of change of a set of data can be used to determine what type of data is represented by the pattern. For example, the rate of change of linear data is constant.}
}

\newglossaryentry{ratio} {
	name=ratio,
	description={A comparison between two quantities, often written in fraction form.}
}

\newglossaryentry{rational expression} {
	name=rational expression,
	description={An expression that can be written as a ratio of two polynomials. The value of the variable cannot make the denominator 0.}
}

\newglossaryentry{rational function} {
	name=rational function,
	description={A function that is expressed as the ratio of two polynomial expressions. The values of the independent variables that make the denominator zero are restricted from the domain.}
}

\newglossaryentry{rational number} {
	name=rational number,
	description={A number that can be written as a ratio of two integers $\frac{a}{b}$ where $b \neq 0$. Their decimal forms are either terminating or repeating. The set of rational numbers is denoted $\Q$. The rational numbers are a subset of the real numbers.}
}

\newglossaryentry{rationalizing the denominator} {
	name=rationalizing the denominator,
	description={The process of making the denominator of a fraction a rational number without changing the value of the expression. It is used to eliminate a radical from the denominator of a fraction.}
}

\newglossaryentry{real number} {
	name=real number,
	description={Denoted $\R$, the set of real numbers include the integers, rational numbers, and irrational numbers, but not imaginary numbers. This is the number set used in Algebra 1. The set is closed under the operations of addition and multiplication. Members can be graphed on the standard number line.  The real numbers is a subset of the complex numbers.}
}

\newglossaryentry{reciprocal} {
	name=reciprocal,
	description={The multiplicative inverse. The reciprocal of a given number is the number it must be multiplied by to get 1 (the multiplicative identity). To find the reciprocal of a number, we can write the number as a fraction and then invert the fraction. The reciprocal of $n$ is $\frac{1}{n}$.}
}

\newglossaryentry{recursive} {
	name=recursive,
	description={Describes a procedure that is applied over and over again, starting with a number or a geometric figure, to produce a sequence of numbers or figures. The procedure requires previous entries in the pattern to find subsequent entries.}
}

\newglossaryentry{recursive rule} {
	name=recursive rule,
	description={Instructions for producing each stage of a sequence from the previous stage. It must contain a description of ``stage 0'', or the starting value.}
}

\newglossaryentry{recursive sequence} {
	name=recursive sequence,
	description={An ordered list of numbers defined by a starting value and a recursive rule. We generate a recursive sequence by applying the rule to the starting value, then applying the rule to the resulting value, and so on.}
}

\newglossaryentry{relatively prime} {
	name=relatively prime,
	description={Two numbers are said to be relatively prime (or coprime) if they have no common factors other than 1. For example, 16 and 21 are relatively prime. In contrast, 21 and 24 are not relatively prime, since both numbers are divisible by 3.}
}

\newglossaryentry{relation} {
	name=relation,
	description={Any set of ordered pairs.}
}

\newglossaryentry{repeating decimal} {
	name=repeating decimal,
	description={A decimal representation of a rational number with a digit or group of digits after the decimal point that repeat infinitely.}
}

\newglossaryentry{root} {
	name=root,
	description={A zero or an $x$-intercept of a function.}
}

\newglossaryentry{sample space} {
	name=sample space,
	description={The set of all possible outcomes of a probability experiment.}
}

\newglossaryentry{scatter plot} {
	name=scatter plot,
	description={A two-variable data display in which values on a horizontal axis represent values of one variable and values on the vertical axis represent values of the other variable. The coordinates of each point represent a pair of data values.}
}

\newglossaryentry{scientific notation} {
	name=scientific notation,
	description={A notation in which a number is written as the product of a number greater than or equal to 1 but less than 10, multiplied by an integer power of 10.}
}

\newglossaryentry{sequence} {
	name=sequence,
	description={A function whose domain is the set of positive integers. A sequence is an ordered list of objects, like numbers. The individual objects are called terms. Unlike a set, order matters, and terms may be repeated.}
}

\newglossaryentry{set} {
	name=set,
	description={An unordered collection of items. Often denoted by listing the elements inside a set of braces.}
}

\newglossaryentry{set notation} {
	name=set notation,
	description={Using curly braces $\{$ and $\}$ to designate quantities that belong to a set. Certain sets do not require the use of braces, as they have symbols used to denote them, like the \gls{null set}, the set of \glspl{integer}, and the set of \glspl{real number}.}
}

\newglossaryentry{simple interest} {
	name=simple interest,
	description={Interest calculated using the formula $I = Prt$. The interest is only ever calculated using the initial investment (called the \gls{principal amount}) and show linear growth.}
}

\newglossaryentry{simplify} {
	name=simplify,
	description={Using algebraic laws and properties which maintain equivalence in order to write an answer so that it fits a set of criteria. The criteria depend on what is being simplified.}
}

\newglossaryentry{simplified radical form} {
	name=simplified radical form,
	description={A radical written so that (1) no perfect square factors exist under the radical (2) no fractions are under the radical and (3) there are no radicals in the 
denominator of the fraction.}
}

\newglossaryentry{slope} {
	name=slope,
	description={The measurement of the steepness of a line, or the rate of change of a linear relationship. Often denoted $m$, and referred to as ``rise over run.'' Given points $(x_1, y_1)$ and $(x_2, y_2)$, the slope of the line between the points is calculated as $m = \frac{\Delta y}{\Delta x} = \frac{y_2-y_1}{x_2-x_1}$.}
}

\newglossaryentry{slope-intercept form} {
	name=slope-intercept form,
	description={The form $y = mx +b$ of a linear equation. The value of $m$ is the slope and the value of $b$ is the $y$-intercept. It is the simplified version of \gls{point-slope form}.}
}

\newglossaryentry{solution} {
	name=solution,
	description={A solution to an equation (or inequality) is any value of the variable (or variables) in the equation (or inequality) that make the equation (or inequality) true. The solution to a system of equations (or inequalities) is the set of all of the points common to all equations in the system. If there is no solution, the system is said to be inconsistent. If there are infinitely many solutions to a system, the system is said to be dependent. If there is a single solution, the system is said to be independent. In a system of two equations in two variables, the solution is the intersection point of the two lines.}
}

\newglossaryentry{solution set} {
	name=solution set,
	description={The set of values that make an equation, inequality, or system true.}
}

%\newglossaryentry{solution to an inequality} {
%	name=solution to an inequality,
%	description={Any value or values of the variable(s) in the inequality that make the inequality true.}
%}

\newglossaryentry{solution set notation} {
	name=solution set notation,
	description={One way to denote the solution set to an equation, written as $S = \{ ~solutions~\}$.}
}

%\newglossaryentry{solution to a system of equations} {
%	name=solution to a system of equations,
%	description={All of the points common to all equations in the system. If there is no solution, the system is said to be inconsistent. If there are infinitely many solutions to a system, the system is said to be dependent. If there is a single solution, the system is said to be independent. In a system of two equations in two variables, the solution is the intersection point of the two lines.}
%}

\newglossaryentry{solve} {
	name=solve,
	description={To find the solution set of an equation.}
}

\newglossaryentry{square root} {
	name=square root,
	description={The square root of a number $a$, denoted $\sqrt{a}$, is the number $b$ such that that $b \cdot b = a$. Every positive number has two square roots, a \gls{principal square root} and a negative square root. The set of real numbers is not closed under the operation of square root.}
}

\newglossaryentry{standard form} {
	name=standard form,
	description={(1) For linear equations, it is an equation of the form $Ax + By = C$, in which $A$ and $B$ are not both 0. (2) For a polynomial, it is an expression written such that it is simplified and the terms are written in decreasing order of degree (highest degree term appears first). (3) For quadratic equations, it is an equation of the form $ax^2 + bx + c$, where $a \neq 0$.}
}

\newglossaryentry{subset} {
	name=subset,
	description={A subset is a set that consists entirely of members from another set. If a set $A$ is a subset of a set $B$, then every item in $A$ is in $B$.}
}

\newglossaryentry{substitution} {
	name=substitution,
	description={To replace a quantity with another one that is equivalent.}
}

\newglossaryentry{substitution method} {
	name=substitution method,
	description={A method for solving a system of equations that involves solving one of the equations for one variable and substituting the resulting expression into the other equation. See also: \gls{elimination method}.}
}

\newglossaryentry{subtraction property of equality} {
	name=subtraction property of equality,
	description={For all real numbers $a$, $b$, and $c$: if $a = b$ then $a - c = b - c$. This property is a restatement of the \gls{addition property of equality} and is used to solve equations.}
}

\newglossaryentry{subtraction property of order} {
	name=subtraction property of order,
	description={For all real numbers $a$, $b$, and $c$: If $a > b$, then $a - c > b - c$. This property is a restatement of the \gls{addition property of order} and is used to solve inequalities.}
}

\newglossaryentry{system of equations} {
	name=system of equations,
	description={A set of two or more equations with the same variables. The equations act as constraints on the variables.}
}

\newglossaryentry{system of inequalities} {
	name=system of inequalities,
	description={A set of two or more inequalities with the same variables. The inequalities act as constraints on the variables.}
}

\newglossaryentry{term} {
	name=term,
	description={An algebraic expression that represents only multiplication and division between variables and constants.}
}

\newglossaryentry{terminating decimal} {
	name=terminating decimal,
	description={A decimal number with a finite number of nonzero digits after the decimal point.}
}

\newglossaryentry{transcendental number} {
	name=transcendental number,
	description={An irrational number that is not algebraic. The number $\pi$ is transcendental because it is not the root of a polynomial equation in one variable with rational coefficients.}
}

\newglossaryentry{trinomial} {
	name=trinomial,
	description={A polynomial with exactly three terms.}
}

\newglossaryentry{two-variable data} {
	name=two-variable data,
	description={A collection of data that measure two traits or quantities. A two-variable data set consists of pairs of values. Compare with: \gls{one-variable data}.}
}

\newglossaryentry{unit rate} {
	name=unit rate,
	description={A ratio in which one of the quantities has the value of 1.}
}

\newglossaryentry{unknown} {
	name=unknown,
	description={A quantity in an equation whose value is not known. In algebra, letters are often used to represent unknowns.}
}

\newglossaryentry{variable} {
	name=variable,
	description={A trait or quantity whose value can change, or vary. In algebra, letters are often used to represent variables.}
}

\newglossaryentry{vector} {
	name=vector,
	description={A quantity that has both a size (or magnitude) and a direction Vectors play an important role in physics and engineering, since many physical quantities (such as velocity, acceleration, and force) are best represented using vectors.}
}

\newglossaryentry{vertex} {
	name=vertex,
	description={Of a parabola, the point where the graph changes direction from increasing to decreasing or from decreasing to increasing.}
}

\newglossaryentry{vertex form} {
	name=vertex form,
	description={A form of a quadratic equation. Given that $(h,k)$ is the vertex, this form is written either as $y-k = a(x-h)^2$ or $y=a(x-h)^2+k$. It can be derived by completing the square on standard form and represents the transformation of $y = ax^2$ by translation $h$ units horizontally and $k$ units vertically.}
}

\newglossaryentry{vertical line test} {
	name=vertical line test,
	description={A method for determining whether a graph on the coordinate plane represents a function. If all possible vertical lines cross the graph only once or not at all, the graph represents a function. If even one vertical line crosses the graph in more than one point, the graph does not represent a function.}
}

\newglossaryentry{vertical motion formula} {
	name=vertical motion formula,
	description={When an object is dropped or launched vertically, its height can be expressed using $h(t) = at^2 + vt + h_0$, where $h(t)$ is the object's height at time $t$, $v$ is its initial vertical velocity, $h_0$ is its starting height, an $a$ is the acceleration of gravity. For dropped objects, $v$ is zero. This formula is used in the study of projectile motion.}
}

\newglossaryentry{vinculum} {
	name=vinculum,
	plural=vincula,
	description={A bar used in mathematics to show grouping. Examples of vincula include: the fraction bar (as in $\frac{1}{x+2}$), the bar used to show repeating digits (as in $0.\overline{3}$), and the horizontal bar of a radical (as in $\sqrt{2+5}$).}
}

\newglossaryentry{x-axis} {
	name=x-axis,
	%sort=x-axis,
	description={The horizontal number line on a coordinate graph. The independent variable is drawn on the $x$-axis.}
}

\newglossaryentry{x-intercept} {
	name=x-intercept,
	%sort=x-intercept,
	description={Any point at which a graph intersects the $x$-axis.}
}

\newglossaryentry{y-axis} {
	name=y-axis,
	%sort=y-axis,
	description={The vertical number line on a coordinate graph. The dependent variable is drawn on the $y$-axis.}
}

\newglossaryentry{y-intercept} {
	name=y-intercept,
	%sort=y-intercept,
	description={Any point at which a graph intersects the $y$-axis.}
}

\newglossaryentry{zero product property} {
	name=zero product property,
	description={Property of real numbers stating that if the product of two or more factors equals zero, then at least one of the factors must equal zero. This property is used along with factoring as a method for solving a polynomial equation.}
}

\newglossaryentry{zero} {
	name=zero,
	description={Of a function, the values of the independent variable that make the corresponding values of the function equal to zero, also known as the \glspl{root} or $x$-intercepts of the function.}
}
