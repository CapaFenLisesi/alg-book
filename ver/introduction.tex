\chapter{Introduction}
\label{ch:introduction}

%\chapquote{Science is operated according to the judicial system. A theory is assumed to be true if there is enough evidence to prove it ``beyond all reasonable doubt''. On the other hand, mathematics does not rely on evidence from fallible experimentation, but it is built on infallible logic.}{Simon Singh, \textit{Fermat's Last Theorem}}

\chapquote{I am always doing that which I cannot do, in order that I may learn how to do it.}{Pablo Picasso, Spanish artist}

The intro is where we describe how our book is different.

%Welcome to your first real mathematics course!
%
%The mathematics courses you have taken before now have focused on arithmetic, which is a branch of mathematics that, for the most part, involves combining numbers by addition, subtraction, multiplication, and division. You have worked hard on developing number sense and perfecting your computational skills with different types of numbers.
%
%In algebra, on the other hand, we will focus our study on mathematical relationships. We will develop the skills needed to describe these relationships abstractly using variables, manipulate them through the use of fundamental laws and properties, and use the patterns and features of these relationships to solve problems.
%
%Learning mathematics is much like learning a language. The best way to learn is to be immersed in the subject and to practice the skills you pick up along the way. We have designed this course with that in mind.
%
%Our approach is based on the idea of a ``function''. Early on, we will discuss what a mathematical function is, and learn different ways to represent functions. As we proceed through the year, we will study specific types, or ``families'', of functions in depth, and learn the various rules, properties, and skills that relate to that family of functions.
%
%We split each of the main units into two components. One component we might call ``mathematical grammar''. Here, we will learn the vocabulary, notation, rules, and properties associated with the big idea of the unit. We will develop sets of tools that will allow us to manipulate algebraic expressions and solve equations. In the second component of each unit we will explore the big ideas of the unit and use the skills we developed earlier to solve problems.
%
%\subsection*{Other ideas...?}
%
%  
%
%I say mention them in this section describing that we have a, this is probably not the right word but, holistic approach to creating a student of mathematics.  
%
%Our appendices can be useful.  Most books have appendices that are related to test prep, but ours could be more like guides for students and teachers, inserts for notebooks, etc.  Is there a way to put links in that portion of the intro to those specific appendices?
%
%Color guide and icon guide for the text boxes and activities...
%
%To include here... or maybe some of these go in an appendix?
%
%\begin{enumerate}
%\item How to be a student of mathematics (tips, responsibilities)?
%\item Expectations (like collaborating in groups)?
%\item How to use the resources found here most effectively?
%\item How to take notes?
%\item Organization?
%\end{enumerate}
%
%A discussion of how problem solving plays a role. ``My dear, all life is a series of problems which we must try and solve: first one, then the next, then the next\ldots{} until at last we die. Why don't you get us an ice cream.'' -- The Dowager Countess of Grantham
